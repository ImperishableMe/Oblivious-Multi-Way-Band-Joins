%======================================================================
% University of Waterloo Thesis Template for LaTeX 
% Last Updated August 2023
% by IST Client Services, 
% University of Waterloo, 200 University Ave. W., Waterloo, Ontario, Canada
% FOR ASSISTANCE, please send mail to ist-helpdesk@uwaterloo.ca

% DISCLAIMER
% To the best of our knowledge, this template satisfies the current uWaterloo thesis requirements.
% However, it is your responsibility to assure that you have met all requirements of the University and your particular department.

% Many thanks for the feedback from many graduates who assisted the development of this template.
% Also note that there are explanatory comments and tips throughout this template.
%======================================================================
% Some important notes on using this template and making it your own...

% The University of Waterloo has required electronic thesis submission since October 2006. 
% See the uWaterloo thesis regulations at
% https://uwaterloo.ca/graduate-studies/thesis.
% This thesis template is geared towards generating a PDF version optimized for viewing on an electronic display, including hyperlinks within the PDF.

% DON'T FORGET TO ADD YOUR OWN NAME AND TITLE in the "hyperref" package configuration below. 
% Search for: PDFTITLE, PDFAUTHOR, PDFSUBJECT, and PDFKEYWORDS.
% THIS INFORMATION GETS EMBEDDED IN THE FINAL PDF DOCUMENT.
% You can view the information if you view properties of the PDF document.

% Many faculties/departments also require one or more printed copies. 
% This template attempts to satisfy both types of output. 
% See additional notes below.
% It is based on the standard "book" document class which provides all necessary sectioning structures and allows multi-part theses.

% If you are using this template in Overleaf (cloud-based collaboration service), then it is automatically processed and previewed for you as you edit.

% For people who prefer to install their own LaTeX distributions on their own computers, and process the source files manually, the following notes provide the sequence of tasks:
 
% E.g. to process a thesis called "mythesis.tex" based on this template, run:

% pdflatex mythesis	-- first pass of the pdflatex processor
% bibtex mythesis	-- generates bibliography from .bib data file(s)
% makeindex         -- should be run only if an index is used 
% pdflatex mythesis	-- fixes numbering in cross-references, bibliographic references, glossaries, index, etc.
% pdflatex mythesis	-- it takes a couple of passes to completely process all cross-references

% If you use the recommended LaTeX editor, Texmaker, you would open the mythesis.tex file, then click the PDFLaTeX button. Then run BibTeX (under the Tools menu).
% Then click the PDFLaTeX button two more times. 
% If you have an index as well,you'll need to run MakeIndex from the Tools menu as well, before running pdflatex
% the last two times.

% N.B. The "pdftex" program allows graphics in the following formats to be included with the "\includegraphics" command: PNG, PDF, JPEG, TIFF
% Tip: Generate your figures and photos in the size you want them to appear in your thesis, rather than scaling them with \includegraphics options.
% Tip: Any drawings you do should be in scalable vector graphic formats: SVG, PNG, WMF, EPS and then converted to PNG or PDF, so they are scalable in the final PDF as well.
% Tip: Photographs should be cropped and compressed so as not to be too large.

% To create a PDF output that is optimized for double-sided printing: 
% 1) comment-out the \documentclass statement in the preamble below, and un-comment the second \documentclass line.
% 2) change the value assigned below to the boolean variable "PrintVersion" from " false" to "true".

%======================================================================
%   D O C U M E N T   P R E A M B L E
% Specify the document class, default style attributes, and page dimensions, etc.
% For hyperlinked PDF, suitable for viewing on a computer, use this:
\documentclass[letterpaper,12pt,titlepage,oneside,final]{book}
 
% For PDF, suitable for double-sided printing, change the PrintVersion variable below to "true" and use this \documentclass line instead of the one above:
%\documentclass[letterpaper,12pt,titlepage,openright,twoside,final]{book}

% Some LaTeX commands I define for my own nomenclature.
% If you have to, it's easier to make changes to nomenclature once here than in a million places throughout your thesis!
\newcommand{\package}[1]{\textbf{#1}} % package names in bold text
\newcommand{\cmmd}[1]{\textbackslash\texttt{#1}} % command name in tt font 
\newcommand{\href}[1]{#1} % does nothing, but defines the command so the print-optimized version will ignore \href tags (redefined by hyperref pkg).
%\newcommand{\texorpdfstring}[2]{#1} % does nothing, but defines the command
% Anything defined here may be redefined by packages added below...

% This package allows if-then-else control structures.
\usepackage{ifthen}
\newboolean{PrintVersion}
\setboolean{PrintVersion}{false}
% CHANGE THIS VALUE TO "true" as necessary, to improve printed results for hard copies by overriding some options of the hyperref package, called below.

% Additional packages for our thesis
\usepackage{algorithm}
\usepackage{algpseudocode} % Modern algorithmic package with \Function, \EndFunction, etc.
\usepackage{booktabs}

%\usepackage{nomencl} % For a nomenclature (optional; available from ctan.org)
\usepackage{amsmath,amssymb,amstext} % Lots of math symbols and environments
\usepackage{amsthm} % For theorem environments
\usepackage{array} % For advanced table column formatting
\usepackage[pdftex]{graphicx} % For including graphics N.B. pdftex graphics driver

% Define theorem environments
\newtheorem{theorem}{Theorem}[chapter]
\newtheorem{lemma}[theorem]{Lemma}
\newtheorem{corollary}[theorem]{Corollary}
\newtheorem{definition}[theorem]{Definition}
\newtheorem{assumption}[theorem]{Assumption}

% Define proof environment settings
\renewcommand{\qedsymbol}{$\square$}

% Load notation and symbol definitions
% ==============================================================================
% NOTATION AND SYMBOL DEFINITIONS
% ==============================================================================
% This file contains all mathematical notation and symbol definitions used
% throughout the thesis. It includes both LaTeX macros and glossary entries.
%
% USAGE CONVENTION:
% - Paragraphs/Text: "name (symbol)" first time, then "name + symbol" after
% - Equations: symbol only (e.g., $\localmult$, $\bigO{...}$)
% - Tables: same as paragraphs (name + symbol format)
%
% Examples:
% - First mention: "local multiplicity ($\localmult$)"
% - Later: "local multiplicity $\localmult$" or "the local multiplicity increases"
% - Equations: "$\localmult = \sum_{i} \alpha_i$"

% ==============================================================================
% LATEX MACROS FOR MATHEMATICAL NOTATION
% ==============================================================================

% ------------------------------------------------------------------------------
% Basic Size and Complexity Notation
% ------------------------------------------------------------------------------
\newcommand{\inputsize}{N} % Total input size
\newcommand{\outputsize}{\text{OUT}} % Output size
\newcommand{\tablesize}{n} % Size of a single table
\newcommand{\numtables}{k} % Number of tables
\newcommand{\bigO}[1]{O(#1)} % Big O notation

% ------------------------------------------------------------------------------
% Algorithm-Specific Notation
% ------------------------------------------------------------------------------
\newcommand{\localmult}{\alpha_{\text{local}}} % Local multiplicity
\newcommand{\finalmult}{\alpha_{\text{final}}} % Final multiplicity
\newcommand{\foreignmult}{\alpha_{\text{foreign}}} % Foreign multiplicity
\newcommand{\cumsum}{C} % Cumulative sum counter

% ------------------------------------------------------------------------------
% Algorithm Variables
% ------------------------------------------------------------------------------
\newcommand{\foreigncumsum}{C_{\text{foreign}}} % Foreign cumulative sum
\newcommand{\localweight}{w_{\text{local}}} % Local weight counter
\newcommand{\localcumsum}{C_{\text{local}}} % Local cumulative sum
\newcommand{\entry}{e} % General entry variable
\newcommand{\startentry}{e_s} % Start entry variable (avoid conflict with constants)
\newcommand{\stopentry}{e_t} % End/terminal entry variable (avoid conflict with constants)
\newcommand{\joinattr}{\text{join\_attr}} % Join attribute

% ------------------------------------------------------------------------------
% Field Accessor Notation
% ------------------------------------------------------------------------------
\newcommand{\fieldtype}{\text{type}} % Entry type field (replaces .type)
\newcommand{\fieldequalitytype}{\text{equality}} % Equality type field
\newcommand{\fielddata}{d} % Entry data/tuple reference field (replaces .data)
\newcommand{\fieldindex}{\text{orig\_idx}} % Original index field

% ------------------------------------------------------------------------------
% Counter Variables  
% ------------------------------------------------------------------------------
\newcommand{\foreignsum}{F_{\text{sum}}} % Foreign sum

% ------------------------------------------------------------------------------
% Auxiliary Variables
% ------------------------------------------------------------------------------
\newcommand{\precedence}{\pi} % Precedence mapping

% ------------------------------------------------------------------------------
% Predefined Complexity Expressions
% ------------------------------------------------------------------------------
\newcommand{\sortcomplexity}{\bigO{\tablesize \log^2 \tablesize}} % Oblivious sorting complexity

% ------------------------------------------------------------------------------
% Algorithm Names and Formatting
% ------------------------------------------------------------------------------
\newcommand{\odbj}{\textsc{odbj}} % ODBJ algorithm name

% ------------------------------------------------------------------------------
% Entry Type Constants
% ------------------------------------------------------------------------------
\newcommand{\typesource}{\text{SOURCE}} % SOURCE entry type constant
\newcommand{\typestart}{\text{START}} % TARGET_START entry type constant
\newcommand{\typeend}{\text{END}} % TARGET_END entry type constant


% ------------------------------------------------------------------------------
% Equality Type Constants
% ------------------------------------------------------------------------------
\newcommand{\equalityequal}{\text{EQ}} % Equal equality type constant
\newcommand{\equalitynonequal}{\text{NEQ}} % Non-equal equality type constant

% ------------------------------------------------------------------------------
% Boundary Parameter Variables
% ------------------------------------------------------------------------------
\newcommand{\deviationone}{d_1} % First boundary deviation
\newcommand{\deviationtwo}{d_2} % Second boundary deviation  
\newcommand{\equalityone}{eq_1} % First boundary equality type
\newcommand{\equalitytwo}{eq_2} % Second boundary equality type
\newcommand{\constraint}{\mathcal{C}} % Constraint function
\newcommand{\constraintparam}{\theta} % Constraint parameters: ((d1, eq1), (d2, eq2))

% ------------------------------------------------------------------------------
% Table Type Terminology (following Krastnikov's ODBJ conventions)
% ------------------------------------------------------------------------------
\newcommand{\inputtable}{input table} % Original unmodified table
\newcommand{\inputtables}{input tables} % Plural form
\newcommand{\augmentedtable}{augmented table} % Table extended with multiplicity metadata
\newcommand{\augmentedtables}{augmented tables} % Plural form
\newcommand{\combinedtable}{combined table} % Array of entries for dual-entry processing
\newcommand{\combinedtables}{combined tables} % Plural form
\newcommand{\expandedtable}{expanded table} % Table where each tuple appears finalmult times
\newcommand{\expandedtables}{expanded tables} % Plural form
\newcommand{\alignedtable}{aligned table} % Expanded table reordered for concatenation
\newcommand{\alignedtables}{aligned tables} % Plural form

% ------------------------------------------------------------------------------
% Table Variable Notation (consistent macro usage)
% ------------------------------------------------------------------------------
\newcommand{\Rcomb}{R_{\text{comb}}} % Combined table variable
\newcommand{\Rtarget}{R_{\text{target}}} % Target table variable  
\newcommand{\Rsource}{R_{\text{source}}} % Source table variable
\newcommand{\Rtruncated}{R_{\text{truncated}}} % Truncated table variable
\newcommand{\Rv}{R_v} % Table at node v
\newcommand{\Rc}{R_c} % Table at child node c

% ------------------------------------------------------------------------------
% Temporary Metadata Field Accessors
% ------------------------------------------------------------------------------
\newcommand{\localsum}{\text{local\_sum}} % Local sum field for cumulative computation
\newcommand{\localinterval}{\text{local\_interval}} % Local interval field for range computation
\newcommand{\foreigninterval}{\text{foreign\_interval}} % Foreign interval field for range computation
\newcommand{\copyindex}{\text{copy\_index}} % Index of copy among all copies of same tuple
\newcommand{\alignmentkey}{\text{alignment\_key}} % Computed alignment position for sorting

% ==============================================================================
% GLOSSARY ENTRIES FOR LIST OF SYMBOLS
% ==============================================================================
% Note: Glossary entries are defined in uw-ethesis.tex after packages are loaded
% to avoid undefined command errors. Only LaTeX macros are defined here.

 

% Hyperlinks make it very easy to navigate an electronic document.
% In addition, this is where you should specify the thesis title and author as they appear in the properties of the PDF document.
% Use the "hyperref" package 
% N.B. HYPERREF MUST BE THE LAST PACKAGE LOADED; ADD ADDITIONAL PKGS ABOVE
\usepackage[pdftex,pagebackref=false]{hyperref} % with basic options
%\usepackage[pdftex,pagebackref=true]{hyperref}
		% N.B. pagebackref=true provides links back from the References to the body text. This can cause trouble for printing.
\hypersetup{
    plainpages=false,       % needed if Roman numbers in frontpages
    unicode=false,          % non-Latin characters in Acrobat’s bookmarks
    pdftoolbar=true,        % show Acrobat’s toolbar?
    pdfmenubar=true,        % show Acrobat’s menu?
    pdffitwindow=false,     % window fit to page when opened
    pdfstartview={FitH},    % fits the width of the page to the window
    pdftitle={Efficient Oblivious Multi-Way Joins with Band Conditions},    % title: CHANGE THIS TEXT!
    pdfauthor={Ruidi Wei},    % author: CHANGE THIS TEXT! and uncomment this line
    pdfsubject={Computer Science},  % subject: CHANGE THIS TEXT! and uncomment this line
    pdfkeywords={oblivious computation} {database joins} {secure computation} {Intel SGX}, % list of keywords, and uncomment this line if desired
    pdfnewwindow=true,      % links in new window
    colorlinks=true,        % false: boxed links; true: colored links
    linkcolor=blue,         % color of internal links
    citecolor=green,        % color of links to bibliography
    filecolor=magenta,      % color of file links
    urlcolor=cyan           % color of external links
}
\ifthenelse{\boolean{PrintVersion}}{   % for improved print quality, change some hyperref options
\hypersetup{	% override some previously defined hyperref options
%    colorlinks,%
    citecolor=black,%
    filecolor=black,%
    linkcolor=black,%
    urlcolor=black}
}{} % end of ifthenelse (no else)

\usepackage[automake,toc,abbreviations]{glossaries-extra} % Exception to the rule of hyperref being the last add-on package
% If glossaries-extra is not in your LaTeX distribution, get it from CTAN (http://ctan.org/pkg/glossaries-extra), 
% although it's supposed to be in both the TeX Live and MikTeX distributions. There are also documentation and 
% installation instructions there.

% Setting up the page margins...
% uWaterloo thesis requirements specify a minimum of 1 inch (72pt) margin at the
% top, bottom, and outside page edges and a 1.125 in. (81pt) gutter margin (on binding side). 
% While this is not an issue for electronic viewing, a PDF may be printed, and so we have the same page layout for both printed and electronic versions, we leave the gutter margin in.
% Set margins to minimum permitted by uWaterloo thesis regulations:
\setlength{\marginparwidth}{0pt} % width of margin notes
% N.B. If margin notes are used, you must adjust \textwidth, \marginparwidth
% and \marginparsep so that the space left between the margin notes and page
% edge is less than 15 mm (0.6 in.)
\setlength{\marginparsep}{0pt} % width of space between body text and margin notes
\setlength{\evensidemargin}{0.125in} % Adds 1/8 in. to binding side of all 
% even-numbered pages when the "twoside" printing option is selected
\setlength{\oddsidemargin}{0.125in} % Adds 1/8 in. to the left of all pages when "oneside" printing is selected, and to the left of all odd-numbered pages when "twoside" printing is selected
\setlength{\textwidth}{6.375in} % assuming US letter paper (8.5 in. x 11 in.) and side margins as above
\raggedbottom

% The following statement specifies the amount of space between paragraphs. Other reasonable specifications are \bigskipamount and \smallskipamount.
\setlength{\parskip}{\medskipamount}

% The following statement controls the line spacing.  
% The default spacing corresponds to good typographic conventions and only slight changes (e.g., perhaps "1.2"), if any, should be made.
\renewcommand{\baselinestretch}{1} % this is the default line space setting

% By default, each chapter will start on a recto (right-hand side) page.
% We also force each section of the front pages to start on a recto page by inserting \cleardoublepage commands.
% In many cases, this will require that the verso (left-hand) page be blank, and while it should be counted, a page number should not be printed.
% The following statements ensure a page number is not printed on an otherwise blank verso page.
\let\origdoublepage\cleardoublepage
\newcommand{\clearemptydoublepage}{%
  \clearpage{\pagestyle{empty}\origdoublepage}}
\let\cleardoublepage\clearemptydoublepage

% Define Glossary terms (This is properly done here, in the preamble and could also be \input{} from a separate file...)
% Main glossary entries -- definitions of relevant terminology
\newglossaryentry{computer}
{
name=computer,
description={A programmable machine that receives input data,
               stores and manipulates the data, and provides
               formatted output}
}

% Nomenclature glossary entries -- New definitions, or unusual terminology
\newglossary*{nomenclature}{Nomenclature}
\newglossaryentry{dingledorf}
{
type=nomenclature,
name=dingledorf,
description={A person of supposed average intelligence who makes incredibly brainless misjudgments}
}

% List of Abbreviations (abbreviations type is built in to the glossaries-extra package)
\newabbreviation{aaaaz}{AAAAZ}{American Association of Amateur Astronomers and Zoologists}

% Note: List of Symbols is now managed in notation.tex

% List of Symbols glossary definition (must be before \makeglossaries)
\newglossary*{symbols}{List of Symbols}

\makeglossaries

% ==============================================================================
% GLOSSARY ENTRIES FOR LIST OF SYMBOLS
% ==============================================================================
% Note: \newglossary* must be before \makeglossaries, but \newglossaryentry must be after

% ------------------------------------------------------------------------------
% Multiplicity Symbols
% ------------------------------------------------------------------------------
\newglossaryentry{localmult}
{
name={$\alpha_{\text{local}}$},
sort={alpha-local},
type=symbols,
description={Local multiplicity: number of times a tuple appears in the subtree rooted at its node}
}

\newglossaryentry{finalmult}
{
name={$\alpha_{\text{final}}$},
sort={alpha-final},
type=symbols,
description={Final multiplicity: number of times a tuple appears in the complete join result}
}

\newglossaryentry{foreignmult}
{
name={$\alpha_{\text{foreign}}$},
sort={alpha-foreign},
type=symbols,
description={Foreign multiplicity: contribution from all tables except the current subtree}
}

% ------------------------------------------------------------------------------
% Algorithm Operation Symbols
% ------------------------------------------------------------------------------
\newglossaryentry{cumsum}
{
name={$C$},
sort={C},
type=symbols,
description={Cumulative sum counter: running total used in dual entry technique}
}

% ------------------------------------------------------------------------------
% Size and Complexity Symbols
% ------------------------------------------------------------------------------
\newglossaryentry{inputsize}
{
name={$N$},
sort={N},
type=symbols,
description={Total input size: sum of sizes of all input tables}
}

\newglossaryentry{outputsize}
{
name={$\text{OUT}$},
sort={OUT},
type=symbols,
description={Output size: number of tuples in the final join result}
}

\newglossaryentry{tablesize}
{
name={$n$},
sort={n},
type=symbols,
description={Size of a single table}
}

\newglossaryentry{numtables}
{
name={$k$},
sort={k},
type=symbols,
description={Number of tables in the multi-way join}
}

% ------------------------------------------------------------------------------
% Tree Notation Symbols
% ------------------------------------------------------------------------------
\newglossaryentry{jointree}
{
name={$T = (V, E)$},
sort={T},
type=symbols,
description={Join tree with table nodes $V$ and join edges $E$}
}

\newglossaryentry{treetotal}
{
name={$\mathcal{T}$},
sort={T-cal},
type=symbols,
description={The entire join tree}
}

\newglossaryentry{subtree}
{
name={$\mathcal{T}_v$},
sort={T-v},
type=symbols,
description={Subtree rooted at node $v$}
}

\newglossaryentry{subtreeminus}
{
name={$\mathcal{T}_v^{-}$},
sort={T-v-minus},
type=symbols,
description={Subtree rooted at $v$ excluding $v$ itself}
}

\newglossaryentry{treeexclude}
{
name={$\mathcal{T} \setminus \mathcal{T}_c$},
sort={T-exclude},
type=symbols,
description={Tree excluding subtree rooted at $c$}
}

% ------------------------------------------------------------------------------
% Additional Field Accessors
% ------------------------------------------------------------------------------
\newglossaryentry{foreignsum}
{
name={$F_{\text{sum}}$},
sort={F-sum},
type=symbols,
description={Foreign multiplicity sum: index position for alignment}
}

\newglossaryentry{localweight}
{
name={$w_{\text{local}}$},
sort={w-local},
type=symbols,
description={Local weight counter: tracks sum of matching tuples' multiplicities}
}

%======================================================================
%   L O G I C A L    D O C U M E N T
% The logical document contains the main content of your thesis.
% Being a large document, it is a good idea to divide your thesis into several files, each one containing one chapter or other significant chunk of content, so you can easily shuffle things around later if desired.
%======================================================================
\begin{document}

%----------------------------------------------------------------------
% FRONT MATERIAL
% title page, examining committee membership (for PhD Thesis only), declaration, borrowers' page, abstract, acknowledgements,
% dedication, table of contents, list of tables, list of figures, nomenclature, etc.
%----------------------------------------------------------------------
% T I T L E   P A G E
% -------------------
% Last updated August 24, 2023, by IST-Client Services
% The title page is counted as page `i' but we need to suppress the
% page number. Also, we don't want any headers or footers.
\pagestyle{empty}
\pagenumbering{roman}

% The contents of the title page are specified in the "titlepage"
% environment.
\begin{titlepage}
        \begin{center}
        \vspace*{1.0cm}

        \Huge
        {\bf Efficient Oblivious Multi-Way Joins with Band Conditions}

        \vspace*{1.0cm}

        \normalsize
        by \\

        \vspace*{1.0cm}

        \Large
        Ruidi Wei \\

        \vspace*{3.0cm}

        \normalsize
        A thesis \\
        presented to the University of Waterloo \\ 
        in fulfillment of the \\
        thesis requirement for the degree of \\
        Master of Mathematics \\
        in \\
        Computer Science \\

        \vspace*{2.0cm}

        Waterloo, Ontario, Canada, 2025 \\

        \vspace*{1.0cm}

        \copyright\ Ruidi Wei 2025 \\
        \end{center}
\end{titlepage}

% The rest of the front pages should contain no headers and be numbered using Roman numerals starting with `ii'
\pagestyle{plain}
\setcounter{page}{2}

\cleardoublepage % Ends the current page and causes all figures and tables that have so far appeared in the input to be printed.
% In a two-sided printing style, it also makes the next page a right-hand (odd-numbered) page, producing a blank page if necessary.
\phantomsection    % allows hyperref to link to the correct page
 
% E X A M I N I N G   C O M M I T T E E (Required for Ph.D. theses only)
% Remove or comment out the lines below to remove this page
\addcontentsline{toc}{chapter}{Examining Committee}
\begin{center}\textbf{Examining Committee Membership}\end{center}
  \noindent
The following served on the Examining Committee for this thesis. The decision of the Examining Committee is by majority vote.
  \bigskip
  
  \noindent
\begin{tabbing}
Internal-External Member: \=  \kill % using longest text to define tab length
External Examiner: \>  Bruce Bruce \\ 
\> Professor, Dept. of Philosophy of Zoology, University of Wallamaloo \\
\end{tabbing} 
  \bigskip
  
  \noindent
\begin{tabbing}
Internal-External Member: \=  \kill % using longest text to define tab length
Supervisor(s): \> Ann Elk \\
\> Professor, Dept. of Zoology, University of Waterloo \\
\> Andrea Anaconda \\
\> Professor Emeritus, Dept. of Zoology, University of Waterloo \\
\end{tabbing}
  \bigskip
  
  \noindent
  \begin{tabbing}
Internal-External Member: \=  \kill % using longest text to define tab length
Internal Member: \> Pamela Python \\
\> Professor, Dept. of Zoology, University of Waterloo \\
\end{tabbing}
  \bigskip
  
  \noindent
\begin{tabbing}
Internal-External Member: \=  \kill % using longest text to define tab length
Internal-External Member: \> Meta Meta \\
\> Professor, Dept. of Philosophy, University of Waterloo \\
\end{tabbing}
  \bigskip
  
  \noindent
\begin{tabbing}
Internal-External Member: \=  \kill % using longest text to define tab length
Other Member(s): \> Leeping Fang \\
\> Professor, Dept. of Fine Art, University of Waterloo \\
\end{tabbing}

\cleardoublepage
\phantomsection    % allows hyperref to link to the correct page

% D E C L A R A T I O N   P A G E
% -------------------------------
  % The following is a sample Declaration Page as provided by the GSPA
  % December 13th, 2006.  It is designed for an electronic thesis.
 \addcontentsline{toc}{chapter}{Author's Declaration}
 \begin{center}\textbf{Author's Declaration}\end{center}

 % Author's Declaration Option ONE - line 118:  
 \noindent
I hereby declare that I am the sole author of this thesis. This is a true copy of the thesis, including any required final revisions, as accepted by my examiners.
  % Author's Declaration Option TWO - line 121. Updated August 21st, 2023. Use the following declaration text if appropriate by removing the percent character and space at the beginning of line 121, and add a percent symbol and space at line 118 to change Author's Declaration Option ONE to a remark that is not printed.
 \noindent  
% This thesis consists of material all of which I authored or co-authored: see Statement of Contributions included in the thesis. This is a true copy of the thesis, including any required final revisions, as accepted by my examiners.
  \bigskip
  
  \noindent
I understand that my thesis may be made electronically available to the public.

\cleardoublepage
\phantomsection    % allows hyperref to link to the correct page

% A B S T R A C T
% ---------------
\addcontentsline{toc}{chapter}{Abstract}
\begin{center}\textbf{Abstract}\end{center}

This thesis introduces the first efficient oblivious algorithm for acyclic multi-way joins with band conditions, extending the classical Yannakakis algorithm to support inequality predicates $(>, <, \geq, \leq)$ without leaking sensitive information through memory access patterns. Band joins, which match tuples over value ranges rather than exact keys, are widely used in temporal, spatial, and proximity-based analytics but present unique challenges in oblivious computation. Our approach employs a novel dual-entry technique that transforms range matching into cumulative sum computations, enabling multiplicity computation in an oblivious manner. The algorithm achieves $O(N\log{N} + OUT \log{OUT})$ complexity, matching state-of-the-art oblivious equality joins while supporting full band constraints. We implement the method in Intel SGX and evaluate it on TPC-H and Twitter datasets, demonstrating practical performance and strong obliviousness guarantees under an honest-but-curious adversary model.

\cleardoublepage
\phantomsection    % allows hyperref to link to the correct page

% A C K N O W L E D G E M E N T S
% -------------------------------
\addcontentsline{toc}{chapter}{Acknowledgements}
\begin{center}\textbf{Acknowledgements}\end{center}

I would like to thank all the little people who made this thesis possible.
\cleardoublepage
\phantomsection    % allows hyperref to link to the correct page

% D E D I C A T I O N
% -------------------
\addcontentsline{toc}{chapter}{Dedication}
\begin{center}\textbf{Dedication}\end{center}

This is dedicated to the one I love.
\cleardoublepage
\phantomsection    % allows hyperref to link to the correct page

% T A B L E   O F   C O N T E N T S
% ---------------------------------
\renewcommand\contentsname{Table of Contents}
\tableofcontents
\cleardoublepage
\phantomsection    % allows hyperref to link to the correct page

% L I S T   O F   F I G U R E S
% -----------------------------
\addcontentsline{toc}{chapter}{List of Figures}
\listoffigures
\cleardoublepage
\phantomsection		% allows hyperref to link to the correct page

% L I S T   O F   T A B L E S
% ---------------------------
\addcontentsline{toc}{chapter}{List of Tables}
\listoftables
\cleardoublepage
\phantomsection		% allows hyperref to link to the correct page

% L I S T   O F   A B B R E V I A T I O N S
% ---------------------------
\renewcommand*{\abbreviationsname}{List of Abbreviations}
\printglossary[type=abbreviations]
\cleardoublepage
\phantomsection		% allows hyperref to link to the correct page

% L I S T   O F   S Y M B O L S
% ---------------------------
\printglossary[type=symbols]
\cleardoublepage
\phantomsection		% allows hyperref to link to the correct page


% Change page numbering back to Arabic numerals
\pagenumbering{arabic}

 

%----------------------------------------------------------------------
% MAIN BODY
% We suggest using a separate file for each chapter of your thesis.
% Start each chapter file with the \chapter command.
% Only use \documentclass or \begin{document} and \end{document} commands in this master document.
% Tip: Putting each sentence on a new line is a way to simplify later editing.
%----------------------------------------------------------------------
%======================================================================
\chapter{Introduction}
%======================================================================

Many applications need joins that are not exact matches but based on ranges. For example, a bank may link transfers that happen within ten minutes to detect fraud, or a hospital may connect lab results taken within a week of a diagnosis. These \emph{band joins} are common in finance, healthcare, and time-based analytics. When such queries are done on sensitive data, organizations often encrypt the data before sending it to the cloud. Encryption hides the contents, but not the way the cloud processes the query. In fact, the pattern of memory accesses itself can leak information---for example, which records are considered ``close'' or how many results are returned. To prevent this leakage, we need algorithms that run \emph{obliviously}, meaning the cloud sees only generic access patterns that reveal nothing about the private data.

For acyclic multi-way joins, the classical Yannakakis algorithm~\cite{yannakakis1981} provides optimal complexity---it evaluates queries in time linear in the input size ($\inputsize$) and output size ($\outputsize$), avoiding the exponential blowup that plagues naive approaches. Recent work has successfully adapted Yannakakis to secure settings, such as the Secure Yannakakis protocol for two-party computation~\cite{wang2021secure}. However, these adaptations handle only equality joins where tuples match on exact values. Band joins present a fundamental new challenge: when matching ranges of values, even the number of matches becomes sensitive information. Consider joining employees with meetings that occurred within their work hours---the access pattern would reveal how many meetings each employee attended, leaking information about their activity level. While generic approaches like Oblivious RAM (ORAM)~\cite{goldreich1996} could hide these patterns, they introduce logarithmic overhead per memory access, with large constant factors that make them impractical for large-scale data processing.

In this thesis, we present the first efficient oblivious algorithm for multi-way band joins. Our approach extends the oblivious Yannakakis framework to handle inequality predicates through a novel dual-entry technique that transforms range matching into cumulative sum computations. We achieve $\bigO{\inputsize \log \inputsize \log \outputsize}$ complexity for acyclic queries, where $\inputsize$ is the input size and $\outputsize$ is the actual output size. This matches the complexity of oblivious equality joins while supporting the full generality of band conditions. We implement our algorithm in Intel SGX~\cite{sgx2016} and demonstrate its practicality on real-world datasets from TPC-H and Twitter.

%----------------------------------------------------------------------
\section{Problem Statement}
%----------------------------------------------------------------------

Our goal is to design an efficient algorithm for evaluating acyclic multi-way joins with band conditions in the oblivious setting. We focus on acyclic queries, which form a large and practical class of queries that can be represented as join trees. Cyclic queries can be transformed to acyclic ones using Generalized Hypertree Decomposition (GHD) obliviously at a cost that becomes impractical for queries with large GHW. In the equality-condition case, the problem is manageable: tuples can be partitioned into \emph{groups} based on the join key, and each group in one table matches exactly one group in another table. This makes it possible to assign the same multiplicity to all tuples in a group without revealing anything sensitive, and also enables techniques like hash joins and oblivious B-trees. Band joins, however, are fundamentally harder. A single group may match to an \emph{entire range} of groups in the other table, and the number of matching groups itself depends on the data. This number is sensitive, so naively accumulating multiplicities across groups would leak information through the access pattern. The challenge is therefore to extend oblivious multi-way join processing beyond equality to support inequality predicates such as $<, >, \leq, \geq$ without leaking information.

%----------------------------------------------------------------------
\section{Contributions}
%----------------------------------------------------------------------

This thesis presents the first oblivious algorithm for acyclic multi-way joins that supports band conditions, extending the classical Yannakakis algorithm to handle inequality predicates ($>, <, \geq, \leq$) while maintaining oblivious access patterns. Our algorithm achieves $\bigO{\inputsize \log \inputsize \log \outputsize}$ complexity for acyclic queries in the oblivious setting, matching the complexity of existing oblivious equality join algorithms while supporting the full generality of range constraints.

At the core of our approach is a novel dual-entry technique for encoding range constraints obliviously. Unlike equality joins where each tuple matches a single group, band joins require matching against ranges of values. Our dual-entry technique transforms this range matching problem into cumulative sum computations that can be performed with data-independent access patterns. We develop a variant of the Yannakakis algorithm that computes actual tuple multiplicities rather than just existence, enabling precise output size determination without leaking information.

To integrate with existing oblivious join frameworks, we design modified bottom-up and top-down passes that are compatible with the \odbj\ framework~\cite{krastnikov2020} while extending its multiplicity computation to support band join conditions. This includes new oblivious expansion and alignment algorithms specifically designed for range-based joins, ensuring that variable-sized outputs from range queries do not reveal sensitive information through access patterns.

We implement our algorithm in Intel SGX and provide a comprehensive experimental evaluation on real-world datasets from TPC-H and Twitter. Our security analysis formally proves that all access patterns remain oblivious throughout the band join processing, ensuring that an adversary observing memory accesses learns nothing about the actual data values or result sizes beyond what is revealed by the public parameters.

%----------------------------------------------------------------------
\section{Thesis Organization}
%----------------------------------------------------------------------

The remainder of this thesis is organized as follows:

\begin{itemize}
\item \textbf{Chapter 2} reviews related work on oblivious joins and identifies the gap our work addresses.
\item \textbf{Chapter 3} provides background on database joins, Yannakakis algorithm, oblivious computation, and secure hardware.
\item \textbf{Chapter 4} presents an overview of our algorithm, developing the approach from binary to multi-way joins.
\item \textbf{Chapter 5} provides the formal algorithm specification with detailed pseudocode and proofs.
\item \textbf{Chapter 6} analyzes the security properties and proves obliviousness.
\item \textbf{Chapter 7} evaluates performance on TPC-H and Twitter datasets.
\item \textbf{Chapter 8} concludes and discusses future work.
\end{itemize}

%======================================================================
\chapter{Related Work}
%======================================================================

This chapter reviews the existing literature on oblivious database operations, focusing on join algorithms. We trace the development from binary equi-joins to our target problem of multi-way band joins, identifying the critical gap that our work addresses.

%----------------------------------------------------------------------
\section{Efficient Oblivious Database Join}
%----------------------------------------------------------------------

Krastnikov et al. proposed the first efficient oblivious algorithm for binary database equi-joins. Their algorithm achieves $\bigO{\tablesize \log^2 \tablesize + \outputsize \log \outputsize}$ complexity where $\tablesize$ is input size and $\outputsize$ is output size, matching the standard non-oblivious sort-merge join up to a logarithmic factor.

The key innovation of \odbj\ is using sorting networks and novel provably-oblivious constructions without relying on ORAM. The algorithm operates in two main phases: multiplicity computation and result construction. During multiplicity computation, tables are combined and sorted by join attribute, with linear passes counting occurrences. The result construction phase uses oblivious distribute and expand operations to create the appropriate number of copies of each tuple.

However, \odbj\ is limited to \textbf{equality predicates only} and \textbf{binary joins} (two tables). It serves as the foundational algorithm for oblivious join processing that we extend in this work.

%----------------------------------------------------------------------
\section{Extension to Band Joins (Inequality Constraints)}
%----------------------------------------------------------------------

Chang et al. made two important extensions to oblivious joins:

\begin{enumerate}
\item \textbf{Binary band joins}: They extended Krastnikov's algorithm to support inequality predicates like $T_1.A \geq T_2.B - c_1$ and $T_1.A \leq T_2.B + c_2$. This maintains oblivious access patterns while handling $>, <, \geq, \leq$ predicates between attributes, but is limited to \textbf{binary joins only}.

\item \textbf{Multiway equi-joins}: They use ORAM-based index nested-loop join with B-tree indices to support joins over multiple tables, but only for \textbf{equality predicates}.
\end{enumerate}

The B-tree approach used for multiway equi-joins cannot be extended to support band conditions. While B-trees are efficient for exact key lookups, range queries in the oblivious setting become problematic---accessing a variable number of nodes for range queries would leak information about the data distribution and result size. To maintain obliviousness, one would need to pad accesses to the worst case, essentially scanning entire tables and negating the benefits of using an index. Therefore, no existing algorithm combines multiple tables with inequality predicates obliviously.

%----------------------------------------------------------------------
\section{Multi-Way Joins (Classical Non-Oblivious)}
%----------------------------------------------------------------------

The classical Yannakakis algorithm achieves optimal $\bigO{\inputsize + \outputsize}$ complexity for acyclic multi-way joins in the non-oblivious setting. It uses a two-phase approach:

\begin{enumerate}
\item \textbf{Bottom-up phase}: Semi-join reductions to eliminate tuples that don't contribute to the final result
\item \textbf{Top-down phase}: Result reconstruction by propagating constraints down the tree
\end{enumerate}

This approach eliminates tuples that don't contribute to the final result, bounding runtime by output size. While Yannakakis achieves optimal complexity for acyclic queries, it is \textbf{not oblivious}---the access patterns reveal information about data distribution and intermediate result sizes. Yannakakis serves as the theoretical foundation for optimal multi-way join processing that we aim to make oblivious.

%----------------------------------------------------------------------
\section{Worst-Case Optimal Join Algorithms}
%----------------------------------------------------------------------

Recent work by Hu and Wu~\cite{hu2025optimal} has made significant progress in developing oblivious algorithms for worst-case optimal multi-way joins, representing an important achievement in oblivious multi-way query processing.

Worst-case optimal algorithms optimize for the theoretical upper bound on output size for a given query structure, assuming maximal matches between tuples regardless of actual data content. This ``worst-case'' bound represents the maximum possible output size that could occur for any instance with the given query and input sizes. In contrast, Yannakakis' algorithm---and our approach building upon it---optimizes for the actual output size of the specific data instance. This output-sensitive approach is particularly beneficial when tuples do not exhibit maximal matching patterns.

Our work thus follows a complementary direction to Hu and Wu's approach. Their worst-case optimal algorithm is particularly valuable for cyclic queries where there is no known efficient method to compute the exact output size. In contrast, for acyclic queries, the exact output size can be efficiently computed, allowing our oblivious Yannakakis-based approach to achieve $\optimalcomplexity$ complexity on acyclic queries, where $\outputsize$ is the actual output size.

%----------------------------------------------------------------------
\section{Critical Gap in the Literature}
%----------------------------------------------------------------------

The existing literature reveals a critical gap: \textbf{No existing solution combines multi-way joins with band conditions obliviously}. 

Table~\ref{tab:related-work-comparison} summarizes the capabilities of existing approaches:

\begin{table}[ht]
\centering
\caption{Comparison of Existing Oblivious Join Approaches}
\label{tab:related-work-comparison}
\begin{tabular}{|p{4cm}|c|c|c|c|}
\hline
\textbf{Approach} & \textbf{Binary} & \textbf{Multi-way} & \textbf{Equality} & \textbf{Band} \\
\hline
\odbj\ (Krastnikov et al.) & \checkmark & & \checkmark & \\
\hline
Chang et al. (binary) & \checkmark & & \checkmark & \checkmark \\
\hline
Opaque/ObliDB & & \checkmark & \checkmark & \\
\hline
Chang et al. (multi-way) & & \checkmark & \checkmark & \\
\hline
Hu and Wu (WCO) & & \checkmark & \checkmark & \\
\hline
\textbf{Our Work} & \checkmark & \checkmark & \checkmark & \checkmark \\
\hline
\end{tabular}
\end{table}

Opaque uses oblivious sort-merge join but is limited to primary-foreign key joins~\cite{opaque2017}. ObliDB supports general multi-way joins using hash join, but this approach essentially computes the Cartesian product, leading to poor performance~\cite{oblidb2020, chang2022}. Hash-based join methods are particularly unsuitable for extension to range queries, as they rely on exact key matching rather than ordering. 

A critical limitation of performing multi-way joins as a series of oblivious binary joins is that it discloses intermediate table sizes, leaking sensitive information about the data distribution and selectivity.

%----------------------------------------------------------------------
\section{Our Approach: Bridging the Gap}
%----------------------------------------------------------------------

Our work bridges this gap by implementing an \textbf{oblivious Yannakakis algorithm} that supports both \textbf{multi-way joins} and \textbf{band conditions}. We achieve this by:

\begin{itemize}
\item Using \textbf{\textsc{odbj} as the base algorithm} for processing neighboring table pairs in the join tree

\item Extending oblivious Yannakakis to support \textbf{inequality predicates} through a novel dual-entry technique

\item Achieving $\optimalcomplexity$ complexity for acyclic queries with full band join support

\item Being the \textbf{first algorithm} to combine efficient oblivious multi-way processing with general range constraints
\end{itemize}

This approach maintains the optimal complexity of Yannakakis (up to logarithmic factors) while supporting the full generality of band conditions, all within the oblivious computation model.

%======================================================================
\chapter{Background}
%======================================================================

This chapter provides the necessary background for understanding our oblivious multi-way join algorithm with band conditions. We cover fundamental database concepts, classical join algorithms including Yannakakis' algorithm, and the principles of oblivious computation and secure hardware.

%----------------------------------------------------------------------
\section{Database Joins and Query Processing}
%----------------------------------------------------------------------

\subsection{Database Join Operations}

A database join is a fundamental operation that combines rows from two or more tables based on a related column between them. The most common type is the equi-join, where rows are matched when they have equal values in specified columns. For example, joining an \texttt{Orders} table with a \texttt{Customers} table on the customer ID creates a result containing order information enriched with customer details.

Join operations form the backbone of relational database queries. In practice, queries often involve multiple tables that need to be joined together---these are called multi-way joins. The order and method of executing these joins significantly impacts query performance, especially as data sizes grow.

\subsection{Join Trees and Query Structure}

Multi-way join queries can be represented as join graphs, where each node represents a table and edges represent join conditions between tables. This tree structure captures the relationships between tables in the query. For instance, in a supply chain query joining \texttt{Suppliers}, \texttt{Parts}, and \texttt{Orders}, the join tree might have \texttt{Parts} at the center, connected to both \texttt{Suppliers} and \texttt{Orders}.

The structure of the join graph determines many properties of the query. When the join graph forms a tree (no cycles), the query is called acyclic. Acyclic queries have special properties that enable more efficient processing algorithms.

\subsection{Acyclic vs Cyclic Queries}

Queries are classified as either acyclic or cyclic based on their join graph structure. Acyclic queries form a tree structure where there is exactly one path between any two tables. This property allows them to be decomposed hierarchically and processed efficiently. For example, a typical business query joining \texttt{Customer} → \texttt{Order} → \texttt{LineItem} → \texttt{Product} forms an acyclic chain.

Cyclic queries contain cycles in their join graph. The classic example is the triangle query where three tables each join with the other two, forming a cycle. For instance, in a social network, finding groups of three people who all know each other requires joining \texttt{Person} with itself three times in a triangular pattern. These cyclic structures prevent direct application of tree-based algorithms like Yannakakis and generally require more complex processing strategies.

\subsection{Handling Cyclic Queries with GHD}

Generalized Hypertree Decomposition (GHD) provides a systematic way to transform cyclic queries into acyclic ones. The key idea is to group relations into ``bags'' arranged in a tree structure, where each bag may contain multiple relations. By pre-computing joins within each bag, we create an acyclic structure that can be processed with tree-based algorithms.

The efficiency of this transformation depends on the Generalized Hypertree Width (GHW) of the query---the minimum number of relations needed in any bag across all possible decompositions. Acyclic queries naturally have GHW = 1 (no grouping needed), while the triangle query has GHW = 2, and a k-cycle has GHW = $\lceil k/2\rceil$. 

The transformation can increase data size exponentially: from N to potentially $N^{\text{GHW}}$, as bags may contain Cartesian products. This exponential blowup makes GHD transformation impractical for queries with large GHW, especially in the oblivious setting where we cannot optimize based on actual data distributions.
%----------------------------------------------------------------------
\section{Band Joins and Range Queries}
%----------------------------------------------------------------------

\subsection{From Equality to Inequality Joins}

While traditional database joins match tuples with exactly equal values, many real-world queries require matching based on ranges or inequalities. These band joins (also called band conditions or range joins) are essential for temporal queries, spatial proximity searches, and interval-based analytics.

Consider a fraud detection query that links credit card transactions occurring within 10 minutes of each other at different locations. This requires joining transactions where the timestamp difference falls within a specified range---a band join rather than an exact match. Similarly, healthcare analytics might join patient visits with lab results taken within a week, or supply chain queries might match orders with shipments arriving within a delivery window.

\subsection{Why Band Joins are Challenging}

Band joins are fundamentally harder than equality joins for several reasons. In an equality join, each value in one table matches at most one group of values in another table. This relationship allows efficient processing using techniques like hash joins or B-tree indexed nested-loop joins.

With band joins, a single value can match an entire range of values in the other table. The number of matches depends on the data distribution---some values might match hundreds of tuples while others match none. This variable fan-out makes it difficult to predict resource requirements and optimize query execution. In the oblivious setting, this challenge is amplified because we cannot allow the access pattern to reveal how many matches each tuple has, as this would leak information about the data distribution.

%----------------------------------------------------------------------
\section{The Yannakakis Algorithm}
%----------------------------------------------------------------------

\subsection{Optimal Processing for Acyclic Queries}

The Yannakakis algorithm~\cite{yannakakis1981}, developed by Mihalis Yannakakis in 1981, provides an elegant solution for evaluating acyclic multi-way joins with optimal complexity. The algorithm achieves $\bigO{\inputsize + \outputsize}$ time, where $\inputsize$ is the total input size and $\outputsize$ is the output size---this is optimal because any algorithm must at least read the input and write the output.

The key insight is to exploit the tree structure of acyclic queries through a two-phase approach. First, a bottom-up pass eliminates tuples that cannot possibly join with tuples in its own subtree. Then, a top-down pass propagates the global constraints to produce the final result. This approach avoids the exponential blowup that can occur with naive join ordering.

\subsection{The Two-Phase Approach}

In the bottom-up phase, the algorithm performs semi-join reductions starting from the leaves of the join tree. Each child table sends information to its parent about which values actually exist, allowing the parent to eliminate tuples that have no matching partners. This process continues up to the root, with each table keeping only tuples that can contribute to the final result based on their subtree.

The top-down phase then propagates constraints from the root back to the leaves. Starting from the filtered root table containing only tuples that exist in the final result, each parent informs its children about which values remain valid in the global context. This ensures that every tuple in the final result participates in the complete join across all tables.

While Yannakakis' algorithm is optimal for non-oblivious settings, it reveals information through its access patterns---which tuples are eliminated and when reveals the selectivity of different join conditions. Our work extends this algorithm to maintain its efficiency while hiding these access patterns.

%----------------------------------------------------------------------
\section{Oblivious Computation}
%----------------------------------------------------------------------

\subsection{The Need for Oblivious Algorithms}

When sensitive data is processed in untrusted environments like public clouds, encryption alone is insufficient. Even with encrypted data, the pattern of memory accesses during computation can leak sensitive information. For example, a binary search reveals the approximate location of the target value through its access pattern, even if all data is encrypted.

Oblivious algorithms address this by ensuring that memory access patterns are independent of the input data. The sequence of memory locations accessed depends only on public parameters like data size, query structure, or a random variable, not on the actual values being processed. This prevents an adversary who can observe all memory accesses from learning anything about the private data.

\subsection{The Oblivious Security Model}

In our security model, we assume an honest-but-curious adversary who can observe all memory access patterns but cannot tamper with the computation. The adversary knows certain public parameters: the sizes of input and output tables, the structure of the join query, and any constants in the join conditions. However, the actual data values, their distribution, and the selectivity of join conditions remain private.

An algorithm is oblivious if two different datasets with the same public parameters produce identical access patterns. This means an adversary watching the memory accesses cannot distinguish between a dataset where two tables are selective and one where the other two tables are selective, as long as the table sizes are the same.

\subsection{Building Blocks for Oblivious Algorithms}

Oblivious algorithms rely on data-independent primitives that operate on tables (arrays of rows) where each row contains values for multiple columns. These primitives ensure access patterns reveal no information about the input data. Table~\ref{tab:oblivious-primitives} summarizes the key primitives used in our algorithm.

\begin{table}[ht]
\centering
\caption{Oblivious Primitives Used in Our Algorithm}
\label{tab:oblivious-primitives}
\begin{tabular}{|>{\raggedright\arraybackslash}p{3.2cm}|>{\raggedright\arraybackslash}p{4.5cm}|>{\raggedright\arraybackslash}p{2.8cm}|>{\raggedright\arraybackslash}p{3cm}|}
\hline
\textbf{Primitive} & \textbf{Description} & \textbf{Runtime} & \textbf{Reference} \\
\hline
Oblivious Sorting & Sorts data using fixed comparison networks independent of input values & $\sortcomplexity$ & Batcher~\cite{batcher1968} \\
\hline
Oblivious Distribution & Moves rows to computed target positions without revealing data patterns & $\bigO{\tablesize \log \tablesize}$ & \odbj~\cite{krastnikov2020} \\
\hline
Oblivious Expansion & Creates multiple copies of rows based on precomputed multiplicities & $\bigO{\tablesize \log \tablesize}$ & \odbj~\cite{krastnikov2020} \\
\hline
Map & Applies an oblivious function that reads from and writes to fixed locations in each row & $\bigO{\tablesize}$ & Standard technique \\
\hline
Linear Scan & Applies an oblivious function to a fixed-size sliding window over a table & $\bigO{\tablesize}$ & Standard technique \\
\hline
Parallel Scan & Applies an oblivious function that takes two rows and operates on fixed locations & $\bigO{\tablesize}$ & Standard technique \\
\hline
\end{tabular}
\end{table}

\textbf{Oblivious Sorting} applies to a table using a sorting order function that takes two rows and returns either $-1$ (first row is ``smaller'') or $1$ (second row is ``smaller''). The algorithm uses fixed comparison networks where the sequence of row comparisons is predetermined based only on the table size, not the actual row values. This ensures that regardless of the data distribution, the same row positions are accessed in the same order, preventing information leakage through access patterns. Our implementation uses Batcher's bitonic sort with $\sortcomplexity$ comparisons for its simplicity and deterministic structure. However, this can be replaced with optimal algorithms like Zig-zag sort~\cite{goodrich2014zigzag} achieving $\bigO{\tablesize \log \tablesize}$ complexity to obtain the theoretical guarantee of $\bigO{\inputsize \log \inputsize \log \outputsize}$ for our overall algorithm, where $\inputsize$ is the input size and $\outputsize$ is the output size.

\textbf{Oblivious Distribution} moves rows to computed target positions within a table without revealing information about where rows are being moved or how many rows end up in each location. This primitive is essential in the \odbj\ framework for repositioning rows according to their multiplicities before expansion. The algorithm uses a series of oblivious sorting and permutation operations to achieve the desired redistribution while ensuring that the access pattern depends only on the table size and target position computation, not on the actual row values or movement patterns.

\textbf{Oblivious Expansion} creates multiple copies of rows based on precomputed multiplicities, ensuring that the duplication process reveals no information about how many copies each row requires. This primitive works in conjunction with oblivious distribution to construct join result tables where each row appears exactly as many times as required by the join semantics. The challenge lies in handling variable expansion factors obliviously---some rows may need many copies while others need few, but the algorithm must access memory in a pattern that depends only on the maximum possible expansion factor, not the actual requirements.

\textbf{Map} applies an oblivious function that reads from and writes to fixed locations within each row of a table. The function operates independently on every row, maintaining data-independent access patterns by accessing predetermined fixed locations within each row. This primitive enables row transformations while preserving obliviousness, such as computing new attributes, applying selection conditions, or reformatting tuple structures. The resulting table may contain modified rows but maintains the same size as the \inputtable.

\textbf{Linear Scan} takes a table and an oblivious function that operates on a fixed number of rows (the window). The function reads from and writes to fixed locations within the window. Linear scan places this fixed-size sliding window on the table and applies the function to each window position as it moves through the table. This maintains data-independent access patterns since the window size and movement are predetermined, ensuring that the memory access sequence depends only on the table size and window size, not on the actual data values encountered.

\textbf{Parallel Scan} takes a table and an oblivious function that operates on two rows (one from each table). The function reads from and writes to fixed locations within these two rows. Parallel scan processes corresponding row pairs with equal indexes from two tables of the same size, applying the oblivious function to each pair sequentially. This maintains data-independent access patterns since the function operates on predetermined fixed locations, ensuring that the memory access sequence depends only on the table sizes and function structure, not on the actual data values.

%----------------------------------------------------------------------
\section{Intel SGX and Secure Hardware}
%----------------------------------------------------------------------

\subsection{Trusted Execution Environments}

Intel Software Guard Extensions (SGX)~\cite{sgx2016} provides hardware-based trusted execution environments called enclaves. These enclaves protect code and data from observation or modification by any external software, including the operating system and hypervisor. When combined with oblivious algorithms, SGX provides end-to-end security for sensitive computations in untrusted environments.

SGX encrypts enclave memory in hardware, ensuring that even physical memory dumps reveal only ciphertext. However, SGX does not hide memory access patterns---the sequence of addresses accessed by the enclave is visible to the OS through page faults and cache effects. This is why oblivious algorithms are essential: they ensure these visible access patterns leak no information about the protected data.

\subsection{Implementing Oblivious Joins in SGX}

Our implementation runs entirely within an SGX enclave, processing encrypted data obliviously. The enclave receives encrypted tables, decrypts them internally, performs the oblivious join computation, and returns encrypted results. Throughout this process, the memory access patterns visible to the untrusted host reveal nothing about the data. The combination of hardware protection and our algorithmic obliviousness provides strong security guarantees against side-channel attacks.

%----------------------------------------------------------------------
\section{Summary}
%----------------------------------------------------------------------

This background establishes the basic concepts underlying our work: the structure and challenges of multi-way band joins, the elegance and optimality of the Yannakakis algorithm for acyclic queries, the principles of oblivious computation for protecting sensitive data, and the role of secure hardware in practical deployments. Building on these foundations, we develop the first oblivious algorithm that combines all these elements---supporting multi-way joins with band conditions while maintaining data-independent access patterns throughout the computation.

%======================================================================
\chapter{Algorithm Overview}
%======================================================================

This chapter provides an intuitive overview of our algorithm before diving into formal specifications. We begin with \odbj's~\cite{krastnikov2020} binary join solution, which separates multiplicity computation from result construction. We then explain how to extend this to multi-way joins by computing multiplicities recursively through tree traversals---a structure that surprisingly mirrors Yannakakis'~\cite{yannakakis1981} classical algorithm. Finally, we introduce our dual-entry technique that enables these computations to work with band conditions, transforming range matching into simple cumulative sums through sorted sequences.

%----------------------------------------------------------------------
\section{From \odbj\ to Oblivious Yannakakis}
%----------------------------------------------------------------------

Our work builds upon recent advances in oblivious database operations, extending binary join techniques to handle multi-way joins with band conditions.

\subsection{Starting with \odbj's Architecture}

Krastnikov et al.'s \odbj~\cite{krastnikov2020} provides an elegant solution for oblivious binary joins, achieving $\bigO{\tablesize \log^2 \tablesize + \outputsize \log \outputsize}$ complexity where $\tablesize$ is input size and $\outputsize$ is output size. The \odbj\ architecture can be separated into two distinct parts: multiplicity computation and result construction.

\subsubsection{Multiplicity Computation Phase}

The algorithm begins by combining both \inputtables\ into a single table sorted by the join attribute, with each tuple tagged by its source table. This combined representation enables counting and recording multiplicities. A forward pass through the sorted table counts occurrences of each unique join key, with two counters tracking tuples from $T_1$ and $T_2$. These counts are then propagated backward to ensure every tuple with the same join key receives the complete count information. 

Through this process, each tuple $(j, d)$ is augmented with two metadata values representing the local multiplicities ($\localmult$): $\alpha_1(j)$, the occurrence of key $j$ in $T_1$, and $\alpha_2(j)$, the occurrence in $T_2$. The significance of these local multiplicity values becomes clear when we consider the join result---each tuple from $T_1$ must appear $\alpha_2(j)$ times (once for each match in $T_2$), while each tuple from $T_2$ must appear $\alpha_1(j)$ times. Thus each tuple obtains its own multiplicity for result construction.

\subsubsection{Result Construction Phase}

With multiplicities computed, \odbj\ constructs the actual join result through three oblivious operations. The \textbf{distribute} operation and the \textbf{expand} operation work together to duplicate each tuple by its multiplicity. Then, the \textbf{align} operation reorders one \expandedtable\ to match the other, ensuring that tuples appear at correct locations, ready to be zipped into the binary join result.

This separation means we must obtain the size of the join result, along with multiplicities of all tuples in the join result, before we can duplicate them for the correct number of times or proceed with any further step. For binary joins, \odbj\ demonstrates this can be done obliviously using only sorting networks and linear scans, avoiding expensive primitives like ORAM~\cite{goldreich1996}.

\subsection{The Multi-Way Multiplicity Challenge}

To extend \odbj~\cite{krastnikov2020} to multi-way joins, we must obtain the multiplicity of each tuple in the full join result before constructing it. For binary joins, \odbj~\cite{krastnikov2020} computes this directly. For multi-way joins over a tree structure, the challenge is: how do we compute the final multiplicity of each tuple when it depends on tables across the entire tree?

We start by looking at a smaller picture, joining the subtree for every table, and call the table tuple's multiplicity in this sub-tree join result ``local multiplicity ($\localmult$)''. For a root tuple, local multiplicity is the same as the final multiplicity ($\finalmult$). This ``local multiplicity of root tuples'' can be computed recursively.

We observe that for an arbitrary parent table tuple, its local multiplicity ($\localmult$) is a product of contributions from joining with each of the child tables. The contribution from each child table is the sum of local multiplicities of matching child table tuples. With a bottom-up traversal of the join tree, we can compute local multiplicities $\localmult$ of all root table tuples.

After obtaining the local multiplicities $\localmult$ of the root table tuples, we view them as final multiplicities ($\finalmult$), and we use a top-down join tree traversal to propagate this final multiplicity $\finalmult$ information across the join tree.

We then perform the distribute and expand, alignment and concatenation phases using the multiplicities.

\subsection{Connection to Yannakakis}

Interestingly, our two-phase structure mirrors Yannakakis's algorithm~\cite{yannakakis1981} for acyclic joins. Yannakakis also uses bottom-up semi-join reduction followed by top-down reconstruction. While Yannakakis computes a boolean value for each tuple indicating whether it exists in the join result or not, we count multiplicities for each tuple indicating how many times it exists in the join result.

%----------------------------------------------------------------------
\section{Band Join Enhancement: Dual Entry Approach}
%----------------------------------------------------------------------

\subsection{The Challenge of Range-Based Multiplicity Computation}

The extension from equality joins to band joins introduces a fundamental challenge in multiplicity computation. Consider two tables A and B with band join condition $A.x \geq B.y - c_1$ and $A.x \leq B.y + c_2$. For a tuple from table A with attribute value $A.x = v$, we must sum the local multiplicities of all tuples from table B that satisfy the range constraint. This differs significantly from equality joins where the matching relationship is one-to-one between groups.

In equality joins where $A.x = B.y$, the multiplicity computation is straightforward: we sort the \combinedtables\ by the join attribute and perform a linear pass, summing tuples with identical values and resetting the sum when the join attribute value changes. This direct accumulation works because each tuple matches exactly those tuples with the same join key value.

Band joins complicate this process because each tuple from table A matches all tuples from table B where $v - c_2 \leq B.y \leq v + c_1$. The challenge lies in efficiently computing the sum of multiplicities across this range without revealing information about the data distribution. A naive approach would require examining each possible matching tuple individually, but this would be inefficient and potentially leak information through access patterns.

\subsection{The Dual Entry Solution}

Our solution transforms the range matching problem into a cumulative sum computation through a dual entry technique. For each tuple $t$ in table A with join attribute value $v$, we create two boundary markers: a start entry at position $v - c_2$ representing the smallest possible matching B value, and an end entry at position $v + c_1$ representing the largest possible matching B value. These boundary markers, combined with the actual tuples from table B, are then sorted by their join attribute values to create a unified sequence.

During a single linear pass through this sorted sequence, we maintain a cumulative sum counter ($\cumsum$) that increments by the local multiplicity ($\localmult$) of each tuple from table B. When we reach the start and end boundary markers for a given tuple from table A, we record the current cumulative sum values. The difference between the end counter and start counter gives precisely the sum of local multiplicities $\localmult$ for all table B tuples that fall within the required range.

This dual entry approach transforms a complex range matching problem into a simple interval computation. The key insight is that start and end entries define interval boundaries in the sorted \combinedtable, and the cumulative counter tracks all relevant contributions seen so far. The difference between consecutive boundary markers captures exactly the multiplicities needed for the range-based join. Crucially, this process remains oblivious since all operations rely solely on oblivious sorting and fixed linear passes with predetermined access patterns, ensuring that no information about the actual data values or match counts is leaked through memory access patterns.



%----------------------------------------------------------------------
\chapter{Detailed Algorithm}
%----------------------------------------------------------------------

%----------------------------------------------------------------------
\section{Algorithm Overview and Notation}
%----------------------------------------------------------------------

Our oblivious multi-way band join algorithm operates on acyclic join trees in four distinct phases, each maintaining data-independent access patterns while computing the complete join result.

\subsection{Algorithm Input and Output}

The algorithm takes as input a join tree $T = (V, E)$ where $V$ are table nodes and $E$ are join edges, along with tables $\{R_1, R_2, \ldots, R_k\}$ where each $R_i$ corresponds to node $v_i \in V$. For each edge $(v_i, v_j) \in E$, band join constraints specify predicates between join attributes. The algorithm produces as output an oblivious join result table $R_{result}$ containing all tuples satisfying the multi-way band join constraints.

\subsection{Table Type Definitions}

Following Krastnikov et al.'s terminology~\cite{krastnikov2020}, we distinguish between different types of tables based on their state in the algorithm:

\begin{itemize}
\item \textbf{\inputtables}: Original unmodified tables $\{R_1, R_2, \ldots, R_k\}$ as provided to the algorithm
\item \textbf{\augmentedtables}: \inputtables\ extended with persistent multiplicity metadata 
\item \textbf{\combinedtables}: Arrays of entries from multiple \augmentedtables\ with temporary metadata, sorted by join attribute for dual-entry processing
\item \textbf{\expandedtables}: \augmentedtables\ where each tuple appears exactly $\finalmult$ times
\item \textbf{\alignedtables}: \expandedtables\ reordered to enable correct concatenation for join result construction
\end{itemize}

\begin{table}[!htbp]
\centering
\caption{Table Schema Evolution Throughout Algorithm Phases}
\label{tab:table-schemas}
\small
\begin{tabular}{|p{2cm}|p{3.5cm}|p{3.5cm}|p{3cm}|}
\hline
\textbf{Type} & \textbf{Original Attrs} & \textbf{Persistent Meta} & \textbf{Temporary Meta} \\
\hline
$R_{input}$ & $\{a_1, a_2, \ldots, a_n\}$ & No & No \\
\hline
$R_{aug}$ & $\{a_1, a_2, \ldots, a_n\}$ & Yes & No \\
\hline
$\Rcomb$ & $\{\fieldtype, \joinattr, \fielddata\}$ & Yes & Yes \\
\hline
$R_{exp}$ & $\{a_1, a_2, \ldots, a_n\}$ & Yes & No \\
\hline
$R_{align}$ & $\{a_1, a_2, \ldots, a_n\}$ & Yes & No \\
\hline
\end{tabular}
\end{table}

\textbf{Metadata presence indicators:}
\begin{itemize}
\item \textbf{Persistent Meta}: Whether table contains metadata that carries forward through phases ($\fieldindex$, $\localmult$, $\finalmult$, $\foreignsum$)
\item \textbf{Temporary Meta}: Whether table contains metadata used only during specific computations. Combined tables use either $\localcumsum$ (bottom-up) or $\foreigncumsum$ (top-down), not both simultaneously
\item \textbf{Note}: Combined tables have a special dual-entry structure where original attributes are transformed into $\{\fieldtype, a, \fielddata\}$ format
\end{itemize}

\subsection{Data Structures and Notation}

The following table summarizes the key data structures, variables, and notation used throughout our oblivious multi-way band join algorithm. The notation distinguishes between entry type constants (using $\tau$ symbols), field accessors for tuple metadata, and various counter variables used in multiplicity computation.

\begin{table}[!htbp]
\centering
\caption{Algorithm Data Structures and Notation}
\label{tab:algorithm-notation}
\small
\begin{tabular}{|p{2.5cm}|p{9.5cm}|}
\hline
\textbf{Notation} & \textbf{Description} \\
\hline
$T = (V, E)$ & Join tree with table nodes $V$ and join edges $E$ \\
\hline
$R_i$ & Relation/table at node $v_i \in V$ \\
\hline
$t$ & Tuple/entry in any table (may include metadata depending on processing phase) \\
\hline
$\localmult$ & Local multiplicity ($\alpha_{\text{local}}$): number of times a tuple appears in subtree join result \\
\hline
$\finalmult$ & Final multiplicity ($\alpha_{\text{final}}$): number of times a tuple appears in complete join result \\
\hline
$\foreignsum$ & Foreign cumulative sum: accumulated foreign contributions from parent multiplicities \\
\hline
% Entry Type Constants
$\typesource$ & SOURCE entry type constant ($\tau_{\text{src}}$) \\
\hline
$\typestart$ & TARGET\_START entry type constant ($\tau_{\text{start}}$) \\
\hline 
$\typeend$ & TARGET\_END entry type constant ($\tau_{\text{end}}$) \\
\hline
% Variables and Data Structures
$\entry$ & General entry variable \\
\hline
$e_s, e_t$ & Start and end entry variables \\
\hline
$\counter$ & Generic counter variable ($C$) \\
\hline
$\foreigncumsum$ & Foreign cumulative sum (temporary): intermediate values during dual-counter computation \\
\hline
$\localweight$ & Local weight counter ($w_{\text{local}}$) \\
\hline
$\copycount$ & Copy counter ($C_{\text{copy}}$) \\
\hline
$\localcumsum$ & Local cumulative sum (temporary): intermediate values during bottom-up computation \\
\hline
$\precedence$ & Entry type precedence mapping ($\pi$) \\
\hline
% Field Accessors
$e.\fieldtype$ & Entry type field (replaces .type) \\
\hline
$e.\fielddata$ & Entry data/tuple reference (replaces .data) \\
\hline
$t.\fieldindex$ & Original tuple index (replaces .orig\_idx) \\
\hline
$t.\joinattr$ & Join attribute ($a$) \\
\hline
$\Rcomb$ & \combinedtable\ of entries for dual-entry processing, sorted by join attribute \\
\hline
$(c_1, c_2)$ & Band join constraint parameters \\
\hline
\end{tabular}
\end{table}

\subsection{Formal Definitions of Multiplicities}

We define three key multiplicities that track tuple participation throughout the join computation:

\textbf{Local Multiplicity ($\localmult$):} For a tuple $t$ in table $\Rv$ at node $v$ in the join tree, the local multiplicity represents the number of times $t$ participates in the join result when considering only the visited portion of the subtree rooted at $v$. During the bottom-up phase, this is computed incrementally: after processing child $c_i$ of $v$, we have:
$$t.\localmult = |\{r \in \bowtie_{T_v^{(i)}} : t \in r\}|$$
where $T_v^{(i)}$ denotes the subtree rooted at $v$ restricted to $v$ itself and its first $i$ processed children (and their subtrees). After all children are processed, $T_v^{(k)} = T_v$ where $k = |children(v)|$. For leaf nodes, $\localmult = 1$ for all tuples.

\textbf{Final Multiplicity ($\finalmult$):} For any tuple $t$ in any table, the final multiplicity represents the number of times $t$ appears in the complete join result across all tables. Formally:
$$t.\finalmult = |\{r \in \bowtie_{T} : t \in r\}|$$
where $T$ is the entire join tree and $\bowtie_{T}$ represents the complete join result. For the root node, $\finalmult = \localmult$. For all other nodes, $\finalmult$ is computed during the top-down phase by propagating information from parent to children.

\textbf{Foreign Multiplicity ($\foreignmult$):} For a tuple $t$ in table $\Rv$ at node $v$ in the join tree, the foreign multiplicity represents the number of times $t$ participates in the join result when considering all tables \emph{outside} the subtree rooted at $v$, plus the node $v$ itself. Formally:
$$t.\foreignmult = |\{r \in \bowtie_{T \setminus T_v^{-}} : t \in r\}|$$
where $T_v^{-}$ denotes the subtree rooted at $v$ excluding $v$ itself, and $T \setminus T_v^{-}$ represents all tables in the tree except those in the children's subtrees. This counts how many times $t$ appears when joining with all tables not in its subtree. The key relationship $\finalmult = \localmult \times \foreignmult$ holds because we assume an acyclic join tree. Specifically, there are no join conditions connecting any node in $T_v^{-}$ to any node in $T \setminus T_v$ (all connections must go through $v$). This independence allows the multiplicities to multiply. In practice, we compute $\foreignmult = \frac{\finalmult}{\localmult}$ during the top-down phase.

\textbf{Foreign Multiplicity Sum ($\foreignsum$):} For a child tuple $t_c$ in table $\Rc$ with parent node $v$, if we were to join all tables in $T \setminus T_c^{-}$ and sort the result by the join attribute between $v$ and $c$, then $t_c.\foreignsum$ is the index of the first entry from the parent table that matches $t_c$. This value is computed during the top-down phase and is used in the align-concatenate phase to determine the correct positioning of tuples in the final result.

\subsection{Common Utilities Across Multiple Phases}

Our algorithm employs several oblivious operations that serve as common utilities across multiple phases.

\textbf{ObliviousSort} utility is the foundation of our approach, utilizing predetermined comparison networks~\cite{batcher1968} to sort tables with fixed access patterns that remain independent of actual data values. The sorting network's structure is determined solely by the input size, ensuring that the sequence of comparisons and swaps follows the same pattern regardless of the data being sorted, which is essential for maintaining oblivious properties in secure computation environments.

\textbf{LinearPass} utility represents our core primitive for processing sorted tables through stateless window operations. This utility applies functions to sliding windows of size 2 over sorted data, where each function operates exclusively on the current window content and position index without any external state dependencies. The function must access (read / write) fixed locations relative to the window, ensuring oblivious access patterns.

\begin{algorithm}[H]
\caption{LinearPass: Apply window function across table with sliding window size 2}
\label{alg:linear-pass}
\begin{algorithmic}[1]
\Function{LinearPass}{$R$, $WindowFunc$}
    \For{$i = 1$ \textbf{to} $|R| - 1$}
        \State $window \leftarrow R[i : i + 1]$ \Comment{Extract window of size 2}
        \State \Call{WindowFunc}{$window$, $i$} \Comment{Apply function to window}
    \EndFor
    \State \Return $R$ \Comment{Return modified table}
\EndFunction
\end{algorithmic}
\end{algorithm}

\textbf{Map} utility provides element-wise transformations across table entries, applying the same function to each row independently. The function reads the input row, and creates an output row with potentially different schema. This is used to change schema of table, adding or removing columns.

\begin{algorithm}[H]
\caption{Map: Apply transformation function to each row independently}
\label{alg:map}
\begin{algorithmic}[1]
\Function{Map}{$R$, $TransformFunc$}
    \State $R_{out} \leftarrow []$ \Comment{Initialize output table}
    \For{$i = 1$ \textbf{to} $|R|$}
        \State $R_{out}[i] \leftarrow$ \Call{TransformFunc}{$R[i]$, $i$}
    \EndFor
    \State \Return $R_{out}$
\EndFunction
\end{algorithmic}
\end{algorithm}

\textbf{ParallelPass} utility processes two tables of same size in parallel by applying a window function to corresponding pairs of rows. The function modifies the rows in-place, similar to LinearPass but operating on aligned pairs from two tables rather than a sliding window.

\begin{algorithm}[H]
\caption{ParallelPass: Apply window function to aligned pairs from two tables}
\label{alg:parallel-pass}
\begin{algorithmic}[1]
\Function{ParallelPass}{$R_1$, $R_2$, $WindowFunc$}
    \Require $|R_1| = |R_2|$ \Comment{Tables must have same size}
    \For{$i = 1$ \textbf{to} $|R_1|$}
        \State $window \leftarrow [R_1[i], R_2[i]]$ \Comment{Create window from aligned pair}
        \State \Call{WindowFunc}{$window$, $i$} \Comment{Apply function to modify in-place}
    \EndFor
    \State \Return $(R_1, R_2)$ \Comment{Return modified tables}
\EndFunction
\end{algorithmic}
\end{algorithm}

\textbf{Additional Primitives:} Our algorithm also relies on two additional oblivious primitives:
\begin{itemize}
\item \textbf{ObliviousExpand:} This primitive from the \odbj\ framework~\cite{krastnikov2020} duplicates each tuple according to its multiplicity, creating an expanded table where each original tuple appears the specified number of times.
\item \textbf{HorizontalConcatenate:} This operation concatenates two tables horizontally, combining all columns from both tables while maintaining the same number of rows. Each row in the result contains the attributes from the corresponding rows in both input tables.
\end{itemize}

\textbf{Join Condition Encoding:} Any join condition between columns can be expressed as an interval constraint. Specifically, a condition between parent column $v.\joinattr$ and child column $c.\joinattr$ can be parsed as: $c.\joinattr \in v.\joinattr + [x, y]$, where the interval $[x, y]$ may use open or closed boundaries and $x, y \in \mathbb{R} \cup \{\pm\infty\}$.

Sample join predicates map to intervals as follows:
\begin{itemize}
\item Equality: $v.\joinattr = c.\joinattr$ maps to $c.\joinattr \in v.\joinattr + [0, 0]$
\item Inequality: $v.\joinattr > c.\joinattr$ maps to $c.\joinattr \in v.\joinattr + (-\infty, 0)$
\item Band constraint: $v.\joinattr \geq c.\joinattr - 1$ maps to $c.\joinattr \in v.\joinattr + [-1, \infty)$
\end{itemize}

When multiple conditions constrain the same join, we compute their interval intersection. For instance, combining $v.\joinattr > c.\joinattr$ (yielding $(-\infty, 0)$) with $v.\joinattr \leq c.\joinattr + 1$ (yielding $[-1, \infty)$) produces the final interval $[-1, 0)$.

The constraint function $\constraint(v,c)$ operationalizes this interval representation by mapping each parent-child relationship to boundary parameters $\constraintparam = ((\deviationone, \equalityone), (\deviationtwo, \equalitytwo))$. Here, $\deviationone$ and $\deviationtwo$ define the interval endpoints, while $\equalityone$ and $\equalitytwo$ specify whether boundaries are closed (EQ) or open (NEQ). This encoding is fundamental to the dual-entry technique used throughout the algorithm. For a target tuple with join attribute value $v$, the boundary parameters create: (i) a START entry at $v + \deviationone$ where if $\equalityone = \text{EQ}$, it includes values $\geq v + \deviationone$, and if $\equalityone = \text{NEQ}$, it includes values $> v + \deviationone$; and (ii) an END entry at $v + \deviationtwo$ where if $\equalitytwo = \text{EQ}$, it includes values $\leq v + \deviationtwo$, and if $\equalitytwo = \text{NEQ}$, it includes values $< v + \deviationtwo$. This encoding allows the dual-entry technique to handle arbitrary range predicates by converting them into boundary entries that can be processed obliviously.

\subsection{Algorithm Structure}

The algorithm begins with initialization to add metadata columns, then operates in four main phases:

\begin{enumerate}
\item \textbf{Initialization (Section~\ref{sec:initialization}):} Add metadata columns to create \augmentedtables
\item \textbf{Phase 1 - Bottom-Up (Section~\ref{sec:bottom-up}):} Compute local multiplicities ($\localmult$) using dual-entry technique for band constraints
\item \textbf{Phase 2 - Top-Down (Section~\ref{sec:top-down}):} Propagate final multiplicities ($\finalmult$) from root to leaves using foreign multiplicity computation
\item \textbf{Phase 3 - Distribution and Expansion (Section~\ref{sec:distribute-expand}):} Create \expandedtables\ by replicating each tuple according to its $\finalmult$ using oblivious distribution
\item \textbf{Phase 4 - Alignment and Concatenation (Section~\ref{sec:align-concat}):} Reorder \expandedtables\ using $\foreignsum$ for alignment, then concatenate to form the final join result
\end{enumerate}

Each phase maintains oblivious access patterns by using the primitives described above. The dual-entry technique transforms range-based band constraints into cumulative sum computations, enabling efficient oblivious processing of inequality joins.

\begin{algorithm}[H]
\caption{Main Algorithm Framework: Oblivious multi-way band join with initialization and four phases}
\label{alg:main}
\begin{algorithmic}[1]
\Function{ObliviousMultiWayBandJoin}{$T = (V, E)$}
    \State $T_{init} \leftarrow$ \Call{InitializeAllTables}{$T$} \Comment{Initialization: Add metadata columns}
    \State $T_{local} \leftarrow$ \Call{BottomUpPhase}{$T_{init}$} \Comment{Phase 1: Compute local multiplicities}
    \State $T_{final} \leftarrow$ \Call{TopDownPhase}{$T_{local}$} \Comment{Phase 2: Compute final multiplicities}
    \State $T_{expanded} \leftarrow$ \Call{DistributeExpand}{$T_{final}$} \Comment{Phase 3: Distribute and expand}
    \State $Result \leftarrow$ \Call{ConstructJoinResult}{$T_{expanded}$, $root$} \Comment{Phase 4: Construct join result}
    \State \Return $Result$ \Comment{Return final join result}
\EndFunction
\end{algorithmic}
\end{algorithm}

%----------------------------------------------------------------------
\section{Initialization}
\label{sec:initialization}
%----------------------------------------------------------------------

The initialization phase prepares the join tree for multiplicity computation by transforming \inputtables\ into \augmentedtables\ with empty metadata columns using the Map primitive. All metadata fields are initialized with null placeholders, and actual values are computed in the bottom-up and top-down phases.

\begin{algorithm}[H]
\caption{Initialize \augmentedtables: Add metadata columns $\{\fieldindex, \localmult, \finalmult, \foreignsum\}$ to input tables using Map primitive with null placeholders. $n_i = |R_i|$, $N = \sum_{i=1}^{k} n_i$.}
\label{alg:initialize}
\begin{algorithmic}[1]
\Function{InitializeAllTables}{$T$}
    \ForAll{nodes $v \in V$}
        \State $R_v \leftarrow$ \Call{Map}{$\Rv$, AddMetadataColumns}
        \State \Call{LinearPass}{$\Rv$, WindowSetOriginalIndex}
    \EndFor
    \State \Return $T$\EndFunction
\end{algorithmic}
\end{algorithm}

\begin{algorithm}[H]
\caption{Add Metadata Columns: Map function to extend tuples with null metadata}
\label{alg:add-metadata}
\begin{algorithmic}[1]
\Function{AddMetadataColumns}{$t$, $index$}
    \State $t.\fieldindex \leftarrow 0$
    \State $t.\localmult \leftarrow null$
    \State $t.\finalmult \leftarrow null$
    \State $t.\foreignsum \leftarrow null$
    \State \Return $t$
\EndFunction
\end{algorithmic}
\end{algorithm}

\begin{algorithm}[H]
\caption{Window Set Original Index: Assign sequential indices with sliding window size 2}
\label{alg:window-set-orig-index}
\begin{algorithmic}[1]
\Function{WindowSetOriginalIndex}{$window$}
    \State $window[1].\fieldindex \leftarrow window[0].\fieldindex + 1$
\EndFunction
\end{algorithmic}
\end{algorithm}

The initialization adds metadata columns using Map, then uses LinearPass to assign sequential original indices. This demonstrates the stateless window-based approach where each tuple's index is computed from its predecessor in the sliding window. 
%----------------------------------------------------------------------
\section{Phase 1: Bottom-Up Multiplicity Computation}
\label{sec:bottom-up}
%----------------------------------------------------------------------

The bottom-up phase computes local multiplicities ($\localmult$) by traversing the join tree $T$ in post-order, as shown in Algorithm~\ref{alg:bottom-up}. For leaf nodes, we initialize each tuple $t \in R_{leaf}$ with $t.\localmult = 1$. For non-leaf nodes, the algorithm processes each parent-child pair $(v, c)$ where $v$ is the parent and $c \in children(v)$. The key insight is that at any point during the traversal, for each visited node $v$, each tuple $t \in \Rv$ has $t.\localmult$ equal to the number of join results it participates in when considering only the portion of the subtree rooted at $v$ that has been visited so far. After all children of $v$ have been processed, $t.\localmult = |\{r \in \bowtie_{T_v^{visited}} : t \in r\}|$ where $T_v^{visited}$ represents the subtree rooted at $v$ restricted to nodes that have been visited in the post-order traversal.

For each parent-child pair $(v, c)$, the algorithm invokes \textsc{ComputeLocalMultiplicities} (Algorithm~\ref{alg:compute-local}) with tables $\Rv$ (target) and $\Rc$ (source), along with constraint parameters $\constraintparam = \constraint(v,c)$ that encode the join condition. This updates each tuple $t_v \in \Rv$ by computing ${t_v}.\localmult^{\text{new}} = {t_v}.\localmult^{\text{old}} \times \sum_{t_c \in \Rc : (t_v, t_c) \text{ satisfy } \constraint(v,c)} t_c.\localmult$, where the second term represents the sum of local multiplicities of all matching tuples from child $c$.

The core innovation lies in the dual-entry technique for handling band join constraints. The \textsc{CombineTable} function (Algorithm~\ref{alg:combine-table}) creates two boundary markers for each tuple in the target (parent) table---START and END entries---that mark where the matching range begins and ends. For example, if a parent tuple with value 10 matches child tuples between values 8 and 12, \textsc{CombineTable} creates a START entry at 8 and an END entry at 12, then combines these boundary entries with the source (child) tuples into a single table.

We then sort by \textsc{ComparatorJoinAttr} (Algorithm~\ref{alg:comparator-join-attr}), which orders entries primarily by join attribute value and secondarily by a precedence based on entry type and equality type. The precedence ordering (defined by \textsc{GetPrecedence} in Algorithm~\ref{alg:get-precedence}) ensures that (START, EQ) and (END, NEQ) entries come first with precedence 1, SOURCE entries have precedence 2, and (START, NEQ) and (END, EQ) entries come last with precedence 3. This careful ordering guarantees that for any target entry $\entry_{target}$ that derives boundary entries $\startentry$ and $\stopentry$, the set of source entries $\{\entry_{source}\}$ appearing between $\startentry$ and $\stopentry$ in the sorted order is exactly the set of source entries that satisfy the join condition with $\entry_{target}$.

We apply \textsc{WindowComputeLocalSum} (Algorithm~\ref{alg:window-compute-local-sum}) via a linear pass to maintain a running sum of local multiplicities: the sum increases by $\localmult$ when we encounter SOURCE entries, and the current sum gets recorded when we hit START/END boundaries. We then sort by \textsc{ComparatorPairwise} to place START and END pairs (which originated from the same target tuple) next to each other. Finally, we apply \textsc{WindowComputeLocalInterval} (Algorithm~\ref{alg:window-compute-local-interval}) via a linear pass to compute the difference between each pair's cumulative sums, yielding the local interval that represents the local multiplicity contribution from the child's subtree for that target tuple.

After creating and sorting the combined table, we apply \textsc{UpdateTargetMultiplicity} (Algorithm~\ref{alg:update-target-multiplicity}) via a parallel pass to propagate the computed intervals back to the parent table, multiplying each target tuple's existing local multiplicity by the contribution from this child (the interval value) to produce the updated local multiplicities.

\begin{algorithm}[H]
\caption{Bottom-Up Phase: Compute local multiplicities from leaves to root}
\label{alg:bottom-up}
\begin{algorithmic}[1]
\Function{BottomUpPhase}{$T$, $root$}
    \State $order \leftarrow$ \Call{PostOrderTraversal}{$T$, $root$}
    \ForAll{nodes $v$ in $order$}
        \If{$v$ is a leaf}
            \ForAll{tuple $t \in R_v$}
                \State $t.\localmult \leftarrow 1$
            \EndFor
        \Else
            \ForAll{child nodes $c$ of $v$}
                \State $R_v \leftarrow$ \Call{ComputeLocalMultiplicities}{$\Rv$, $\Rc$, $\constraint(v,c)$}
            \EndFor
        \EndIf
    \EndFor
    \State \Return $T$\EndFunction
\end{algorithmic}
\end{algorithm}

\begin{algorithm}[H]
\caption{Post-Order Traversal: Visit children before parents in tree}
\label{alg:post-order}
\begin{algorithmic}[1]
\Function{PostOrderTraversal}{$T$, $root$}
    \State $order \leftarrow$ empty list
    \ForAll{child nodes $c$ of $root$}
        \State $order \leftarrow order + $ \Call{PostOrderTraversal}{$T$, $c$}
    \EndFor
    \State Append $root$ to $order$
    \State \Return $order$
\EndFunction
\end{algorithmic}
\end{algorithm}

\begin{algorithm}[H]
\caption{Compute Local Multiplicities: Compute new local multiplicities for parent node in bottom-up phase}
\label{alg:compute-local}
\begin{algorithmic}[1]
\Function{ComputeLocalMultiplicities}{$\Rtarget$, $\Rsource$, $\constraintparam$}
    \State $\Rcomb \leftarrow$ \Call{CombineTable}{$\Rtarget$, $\Rsource$, $\constraintparam$}
    \State $\Rcomb \leftarrow$ \Call{Map}{$\Rcomb$, $\lambda e: (e.\localsum \leftarrow e.\localmult, e.\localinterval \leftarrow 0, e)$}
    \State \Call{ObliviousSort}{$\Rcomb$, ComparatorJoinAttr}
    \State \Call{LinearPass}{$\Rcomb$, WindowComputeLocalSum}
    \State \Call{ObliviousSort}{$\Rcomb$, ComparatorPairwise}
    \State \Call{LinearPass}{$\Rcomb$, WindowComputeLocalInterval}
    \State \Call{ObliviousSort}{$\Rcomb$, ComparatorEndFirst}
    \State $\Rtruncated \leftarrow \Rcomb[1:|\Rtarget|]$
    \State \Call{ParallelPass}{$\Rtruncated$, $\Rtarget$, UpdateTargetMultiplicity}
    \State \Return $\Rtarget$
\EndFunction
\end{algorithmic}
\end{algorithm}

\begin{algorithm}[H]
\caption{Combine Table: Create start/end boundary entries for each target tuple and merge with source entries}
\label{alg:combine-table}
\begin{algorithmic}[1]
\Function{CombineTable}{$\Rtarget$, $\Rsource$, $\constraintparam$}
    \State $((\deviationone, \equalityone), (\deviationtwo, \equalitytwo)) \leftarrow \constraintparam$
    \State $R_{source}' \leftarrow$ \Call{Map}{$\Rsource$, \textbf{function}($t$):}
    \State \quad $e.\fieldtype \leftarrow \typesource$
    \State \quad $e.\fieldequalitytype \leftarrow null$
    \State \quad $e.\joinattr \leftarrow t.\joinattr$
    \State \quad $e.\fieldindex \leftarrow t.\fieldindex$
    \State \quad $e.\localmult \leftarrow t.\localmult$
    \State \quad $e.\finalmult \leftarrow t.\finalmult$
    \State \quad $e.\foreignsum \leftarrow t.\foreignsum$
    \State \quad \Return $e$
    \State $R_{begin}' \leftarrow$ \Call{Map}{$\Rtarget$, \textbf{function}($t$):}
    \State \quad $e.\fieldtype \leftarrow \typestart$
    \State \quad $e.\fieldequalitytype \leftarrow \equalityone$
    \State \quad $e.\joinattr \leftarrow t.\joinattr + \deviationone$
    \State \quad $e.\fieldindex \leftarrow t.\fieldindex$
    \State \quad $e.\localmult \leftarrow t.\localmult$
    \State \quad $e.\finalmult \leftarrow t.\finalmult$
    \State \quad $e.\foreignsum \leftarrow t.\foreignsum$
    \State \quad \Return $e$
    \State $R_{end}' \leftarrow$ \Call{Map}{$\Rtarget$, \textbf{function}($t$):}
    \State \quad $e.\fieldtype \leftarrow \typeend$
    \State \quad $e.\fieldequalitytype \leftarrow \equalitytwo$
    \State \quad $e.\joinattr \leftarrow t.\joinattr + \deviationtwo$
    \State \quad $e.\fieldindex \leftarrow t.\fieldindex$
    \State \quad $e.\localmult \leftarrow t.\localmult$
    \State \quad $e.\finalmult \leftarrow t.\finalmult$
    \State \quad $e.\foreignsum \leftarrow t.\foreignsum$
    \State \quad \Return $e$
    \State $\Rcomb \leftarrow R_{source}' + R_{begin}' + R_{end}'$
    \State \Return $\Rcomb$
\EndFunction
\end{algorithmic}
\end{algorithm}

\begin{algorithm}[H]
\caption{Comparator Join Attribute: Sort entries by join attribute, then by entry type precedence\\
\small\textit{Precedence: (START, EQ) → 1, (END, NEQ) → 1, (SOURCE, null) → 2, (START, NEQ) → 3, (END, EQ) → 3}}
\label{alg:comparator-join-attr}
\begin{algorithmic}[1]
\Function{ComparatorJoinAttr}{$e_1$, $e_2$}
    \If{$e_1.\joinattr < e_2.\joinattr$}
        \Return -1
    \ElsIf{$e_1.\joinattr > e_2.\joinattr$}
        \Return 1
    \Else
        \State $p_1 \leftarrow$ \Call{GetPrecedence}{$(e_1.\fieldtype, e_1.\fieldequalitytype)$}
        \State $p_2 \leftarrow$ \Call{GetPrecedence}{$(e_2.\fieldtype, e_2.\fieldequalitytype)$}
        \If{$p_1 < p_2$}
            \Return -1
        \ElsIf{$p_1 > p_2$}
            \Return 1
        \Else
            \Return 0
        \EndIf
    \EndIf
\EndFunction
\end{algorithmic}
\end{algorithm}

\begin{algorithm}[H]
\caption{Window Compute Local Sum: Compute cumulative sum with sliding window size 2}
\label{alg:window-compute-local-sum}
\begin{algorithmic}[1]
\Function{WindowComputeLocalSum}{$window$}
    \If{$window[1].\fieldtype = \typesource$}
        \State $window[1].\localsum \leftarrow window[0].\localsum + window[1].\localmult$
    \Else  \Comment{$window[1].\fieldtype \in \{\typestart, \typeend\}$}
        \State $window[1].\localsum \leftarrow window[0].\localsum$
    \EndIf
\EndFunction
\end{algorithmic}
\end{algorithm}

\begin{algorithm}[H]
\caption{Comparator Pairwise: Organize entries for pairwise START/END processing by grouping targets first, then by index}
\label{alg:comparator-pairwise}
\begin{algorithmic}[1]
\Function{ComparatorPairwise}{$e_1$, $e_2$}
    \Comment{First: Target entries (START/END) before SOURCE entries}
    \If{$e_1.\fieldtype \in \{\typestart, \typeend\}$ and $e_2.\fieldtype = \typesource$}
        \Return -1
    \ElsIf{$e_1.\fieldtype = \typesource$ and $e_2.\fieldtype \in \{\typestart, \typeend\}$}
        \Return 1
    \EndIf
    \Comment{Second: Sort by original index}
    \If{$e_1.\fieldindex < e_2.\fieldindex$}
        \Return -1
    \ElsIf{$e_1.\fieldindex > e_2.\fieldindex$}
        \Return 1
    \EndIf
    \Comment{Third: START before END for same index}
    \If{$e_1.\fieldtype = \typestart$ and $e_2.\fieldtype = \typeend$}
        \Return -1
    \ElsIf{$e_1.\fieldtype = \typeend$ and $e_2.\fieldtype = \typestart$}
        \Return 1
    \Else
        \Return 0
    \EndIf
\EndFunction
\end{algorithmic}
\end{algorithm}

\begin{algorithm}[H]
\caption{Window Compute Local Interval: Compute range difference between start/end entries with window size 2}
\label{alg:window-compute-local-interval}
\begin{algorithmic}[1]
\Function{WindowComputeLocalInterval}{$window$}
    \If{$window[0].\fieldtype = \typestart$ and $window[1].\fieldtype = \typeend$}
        \State $window[1].\localinterval \leftarrow window[1].\localsum - window[0].\localsum$
    \EndIf
\EndFunction
\end{algorithmic}
\end{algorithm}

\begin{algorithm}[H]
\caption{Comparator End First: Put END entries first, then sort by original index}
\label{alg:comparator-end-first}
\begin{algorithmic}[1]
\Function{ComparatorEndFirst}{$e_1$, $e_2$}
    \Comment{First: END entries before all others}
    \If{$e_1.\fieldtype = \typeend$ and $e_2.\fieldtype \neq \typeend$}
        \Return -1
    \ElsIf{$e_1.\fieldtype \neq \typeend$ and $e_2.\fieldtype = \typeend$}
        \Return 1
    \EndIf
    \Comment{Second: Sort by original index}
    \If{$e_1.\fieldindex < e_2.\fieldindex$}
        \Return -1
    \ElsIf{$e_1.\fieldindex > e_2.\fieldindex$}
        \Return 1
    \Else
        \Return 0
    \EndIf
\EndFunction
\end{algorithmic}
\end{algorithm}

\begin{algorithm}[H]
\caption{Update Target Multiplicity: Multiply target's local multiplicity by computed interval}
\label{alg:update-target-multiplicity}
\begin{algorithmic}[1]
\Function{UpdateTargetMultiplicity}{$e_{combined}$, $e_{target}$}
    \State $e_{target}.\localmult \leftarrow e_{target}.\localmult \times e_{combined}.\localinterval$
\EndFunction
\end{algorithmic}
\end{algorithm}

\begin{algorithm}[H]
\caption{Get Entry Type Precedence: Map (entry\_type, equality\_type) tuple to precedence value}
\label{alg:get-precedence}
\begin{algorithmic}[1]
\Function{GetPrecedence}{$(entry\_type, equality\_type)$}
    \If{$(entry\_type, equality\_type) = (\typestart, \equalityequal)$}
        \Return 1
    \ElsIf{$(entry\_type, equality\_type) = (\typeend, \equalitynonequal)$}
        \Return 1
    \ElsIf{$(entry\_type, equality\_type) = (\typesource, null)$}
        \Return 2
    \ElsIf{$(entry\_type, equality\_type) = (\typestart, \equalitynonequal)$}
        \Return 3
    \ElsIf{$(entry\_type, equality\_type) = (\typeend, \equalityequal)$}
        \Return 3
    \EndIf
\EndFunction
\end{algorithmic}
\end{algorithm}




%----------------------------------------------------------------------
\section{Phase 2: Top-Down Final Multiplicity Propagation}
\label{sec:top-down}
%----------------------------------------------------------------------

The top-down phase propagates final multiplicities ($\finalmult$) from the root to all nodes in the tree, mirroring the reconstruction phase of Yannakakis~\cite{yannakakis1981}. This phase computes how many times each tuple appears in the complete join result by considering contributions from outside its subtree. The traversal proceeds in pre-order, starting from the root where $\finalmult = \localmult$ (since the root has no ancestors), then propagating downward to compute each child's final multiplicity based on its parent's values.

For each parent-child pair $(v, c)$ during the pre-order traversal, the algorithm invokes \textsc{PropagateFinalMultiplicities} (Algorithm~\ref{alg:propagate-final}) to compute the final multiplicities for child table $\Rc$. The key insight is that each child tuple's final multiplicity equals its local multiplicity times its foreign multiplicity, where the foreign multiplicity ($\foreignmult$) represents the number of join results from tables outside the child's subtree that connect through the parent. This is computed as: $t_c.\finalmult = t_c.\localmult \times t_c.\foreignmult$.

The core question in the top-down phase is: what would be the multiplicity of each parent tuple if we excluded the child table and its entire subtree? That is, what is the multiplicity of parent (source) table entries in the join result of $\mathcal{T} \setminus \mathcal{T}_c$? Since the final multiplicity is the product of contributions from all neighbors, we can recover this by division. We use a running sum called "local weight" to track the sum of matching child tuples' local multiplicities---this represents the child subtree's contribution. By dividing a parent tuple's final multiplicity by this local weight, we recover its multiplicity in $\mathcal{T} \setminus \mathcal{T}_c$. The sum of these multiplicities for all matching parent tuples gives us the foreign multiplicity ($\foreignmult$), which represents the contribution from $\mathcal{T} \setminus \mathcal{T}_c$ and complements the local multiplicity (contribution from $\mathcal{T}_c$).

To compute these values obliviously, we employ a similar structure as the bottom-up phase. We use \textsc{CombineTable} to create START and END boundaries for target table tuples, while SOURCE entries represent source table tuples. The difference from bottom-up is that here the child table is the target (receiving multiplicities) and the parent table is the source (providing multiplicities). After sorting by \textsc{ComparatorJoinAttr} (Algorithm~\ref{alg:comparator-join-attr}), we apply \textsc{WindowComputeForeignSum} (Algorithm~\ref{alg:window-compute-foreign-sum}) via a linear pass that simultaneously tracks two counters. When we encounter START/END boundaries, we update the local weight by adding or subtracting the child tuple's local multiplicity. When we encounter SOURCE entries (parent tuples), we increment the foreign cumulative sum by the parent's final multiplicity divided by the current local weight. This division recovers the parent's multiplicity in $\mathcal{T} \setminus \mathcal{T}_c$, and the accumulation gives each child tuple its foreign multiplicity sum ($\foreignsum$). This $\foreignsum$ serves dual purposes: it provides the foreign multiplicity for computing $\finalmult = \localmult \times \foreignmult$, and later serves as the alignment key during result construction.

After processing all parent-child pairs in pre-order, every tuple in every table has its final multiplicity computed, representing exactly how many times it will appear in the complete join result. This prepares the tables for the distribution and expansion phase where tuples are replicated according to their final multiplicities.

\begin{algorithm}[H]
\caption{Top-Down Phase: Propagate final multiplicities from root to leaves}
\label{alg:top-down}
\begin{algorithmic}[1]
\Function{TopDownPhase}{$T$, $root$}
    \ForAll{tuple $t \in R_{root}$}
        \State $t.\finalmult \leftarrow t.\localmult$ \Comment{Root final = local}
    \EndFor
    \ForAll{nodes $v$ in pre-order traversal of $T$ from $root$}
        \ForAll{child nodes $c$ of $v$}
            \State $R_c \leftarrow$ \Call{PropagateFinalMultiplicities}{$\Rv$, $\Rc$, $\constraint(v,c)$}
        \EndFor
    \EndFor
    \State \Return $T$ \Comment{Return tree with tables containing computed final multiplicities}
\EndFunction
\end{algorithmic}
\end{algorithm}

\begin{algorithm}[H]
\caption{Propagate Final Multiplicities: Distribute parent multiplicities to children using dual counters}
\label{alg:propagate-final}
\begin{algorithmic}[1]
\Function{PropagateFinalMultiplicities}{$\Rsource$, $\Rtarget$, $\constraintparam$}
    \State $\Rcomb \leftarrow$ \Call{CombineTable}{$\Rtarget$, $\Rsource$, $\constraintparam$}
    \State $\Rcomb \leftarrow$ \Call{Map}{$\Rcomb$, $\lambda e:$}
    \State \quad $(e.\localweight \leftarrow e.\localmult,$
    \State \quad $\phantom{(}e.\foreigncumsum \leftarrow 0,$
    \State \quad $\phantom{(}e.\foreigninterval \leftarrow 0, e)$
    \State \Call{ObliviousSort}{$\Rcomb$, ComparatorJoinAttr}
    \State \Call{LinearPass}{$\Rcomb$, WindowComputeForeignSum}
    \State \Call{ObliviousSort}{$\Rcomb$, ComparatorPairwise}  
    \State \Call{LinearPass}{$\Rcomb$, WindowComputeForeignInterval}
    \State \Call{ObliviousSort}{$\Rcomb$, ComparatorEndFirst}
    \State $\Rtruncated \leftarrow \Rcomb[1:|\Rtarget|]$
    \State \Call{ParallelPass}{$\Rtruncated$, $\Rtarget$, UpdateTargetFinalMultiplicity}
    \State \Return $\Rtarget$
\EndFunction
\end{algorithmic}
\end{algorithm}

\begin{algorithm}[H]
\caption{Window Compute Foreign Sum: Track foreign and local weight counters simultaneously}
\label{alg:window-compute-foreign-sum}
\begin{algorithmic}[1]
\Function{WindowComputeForeignSum}{$window$}
    \If{$window[1].\fieldtype = \typestart$}
        \State $window[1].\localweight \leftarrow window[0].\localweight + window[1].\localmult$
        \State $window[1].\foreigncumsum \leftarrow window[0].\foreigncumsum$
    \ElsIf{$window[1].\fieldtype = \typeend$}
        \State $window[1].\localweight \leftarrow window[0].\localweight - window[1].\localmult$
        \State $window[1].\foreigncumsum \leftarrow window[0].\foreigncumsum$
    \ElsIf{$window[1].\fieldtype = \typesource$}
        \State $window[1].\localweight \leftarrow window[0].\localweight$
        \State $window[1].\foreigncumsum \leftarrow window[0].\foreigncumsum + window[1].\finalmult / window[1].\localweight$
    \EndIf
\EndFunction
\end{algorithmic}
\end{algorithm}

\begin{algorithm}[H]
\caption{Window Compute Foreign Interval: Compute foreign multiplicity from START/END cumulative sums}
\label{alg:window-compute-foreign-interval}
\begin{algorithmic}[1]
\Function{WindowComputeForeignInterval}{$window$}
    \If{$window[0].\fieldtype = \typestart$ and $window[1].\fieldtype = \typeend$}
        \State $\foreigninterval \leftarrow window[1].\foreigncumsum - window[0].\foreigncumsum$
        \State $window[1].\foreigninterval \leftarrow \foreigninterval$
        \State $window[1].\foreignsum \leftarrow window[0].\foreigncumsum$ \Comment{Record alignment position}
    \EndIf
\EndFunction
\end{algorithmic}
\end{algorithm}

\begin{algorithm}[H]
\caption{Update Target Final Multiplicity: Propagate foreign intervals to compute final multiplicities}
\label{alg:update-target-final}
\begin{algorithmic}[1]
\Function{UpdateTargetFinalMultiplicity}{$e$, $t$}
    \State $t.\finalmult \leftarrow e.\foreigninterval \times t.\localmult$
    \State $t.\foreignsum \leftarrow e.\foreignsum$ \Comment{For alignment}
\EndFunction
\end{algorithmic}
\end{algorithm}

%----------------------------------------------------------------------
\section{Phase 3: Distribution and Expansion}
\label{sec:distribute-expand}
%----------------------------------------------------------------------

Each tuple must be replicated according to its final multiplicity $\finalmult$. We use the oblivious distribute-and-expand technique from \odbj~\cite{krastnikov2020}, which creates exactly $\finalmult$ copies of each tuple while maintaining oblivious access patterns. This technique first distributes tuples to their target positions, then expands them to fill the required space. The key property is that the expansion is data-oblivious: the access pattern depends only on the multiplicities, not on the actual data values.

%----------------------------------------------------------------------
\section{Phase 4: Alignment and Concatenation}
\label{sec:align-concat}
%----------------------------------------------------------------------

After expansion, tables must be aligned so that matching tuples appear in the same rows. The parent table is sorted by join attributes (and secondarily by other attributes for deterministic ordering), creating groups of identical tuples. Each group represents a distinct combination from the parent table that will be matched with corresponding child tuples.

The child table alignment uses the formula $\foreignsum + (\copyindex \div \localmult)$, where:
\begin{itemize}
\item $\foreignsum$ is the index of the first parent group that matches this child tuple
\item $\copyindex$ is the index of this copy among all copies of the same original tuple (0 to $\finalmult-1$)
\item $\localmult$ is the child tuple's local multiplicity
\end{itemize}

This formula ensures that every $\localmult$ copies of a child tuple increment to the next parent group, correctly distributing child copies across matching parent groups. After sorting by this alignment key, corresponding rows from parent and child tables are horizontally concatenated to form the partial join result. This process continues recursively through the join tree until all tables are combined.

\begin{algorithm}[H]
\caption{Result Construction}
\label{alg:result-construction}
\begin{algorithmic}[1]
\Function{ConstructJoinResult}{$T$, $root$}
    \State $result \leftarrow$ \Call{ObliviousExpand}{$R_{root}$} \Comment{Expand root table}
    \ForAll{nodes $v$ in pre-order traversal of $T$ from $root$}
        \ForAll{child nodes $c$ of $v$}
            \State $R_c^{expanded} \leftarrow$ \Call{ObliviousExpand}{$\Rc$} \Comment{Expand child table}
            \State $result \leftarrow$ \Call{AlignAndConcatenate}{$result$, $R_c^{expanded}$}
        \EndFor
    \EndFor
    \Return $result$
\EndFunction
\end{algorithmic}
\end{algorithm}

\begin{algorithm}[H]
\caption{Align and Concatenate}
\label{alg:align-concatenate}
\begin{algorithmic}[1]
\Function{AlignAndConcatenate}{$R_{accumulator}$, $R_{child}$}
    \State \Call{ObliviousSort}{$R_{accumulator}$, JoinThenOtherAttributes} \Comment{Sort by join attrs, then others}
    \State \Call{LinearPass}{$R_{child}$, ComputeAlignmentKey} \Comment{Set alignment key for each tuple}
    \State \Call{ObliviousSort}{$R_{child}$, AlignmentKeyComparator}
    \State \Return \Call{HorizontalConcatenate}{$R_{accumulator}$, $R_{child}$}
\EndFunction
\end{algorithmic}
\end{algorithm}

\begin{algorithm}[H]
\caption{Compute Alignment Key}
\label{alg:compute-alignment}
\begin{algorithmic}[1]
\Function{ComputeAlignmentKey}{$tuple$}
    \State $tuple.\alignmentkey \leftarrow tuple.\foreignsum + (tuple.\copyindex \div tuple.\localmult)$
\EndFunction
\end{algorithmic}
\end{algorithm}

\begin{algorithm}[H]
\caption{Join Then Other Attributes Comparator}
\label{alg:join-then-other-comparator}
\begin{algorithmic}[1]
\Function{JoinThenOtherAttributes}{$t_1$, $t_2$}
    \If{$t_1.\joinattr < t_2.\joinattr$}
        \Return -1
    \ElsIf{$t_1.\joinattr > t_2.\joinattr$}
        \Return 1
    \Else
        \Comment{Compare by parent's original index (maintained in $R_{accumulator}$)}
        \Return \Call{CompareOriginalIndex}{$t_1$, $t_2$}
    \EndIf
\EndFunction
\end{algorithmic}
\end{algorithm}

\begin{algorithm}[H]
\caption{Alignment Key Comparator}
\label{alg:alignment-comparator}
\begin{algorithmic}[1]
\Function{AlignmentKeyComparator}{$t_1$, $t_2$}
    \If{$t_1.\alignmentkey < t_2.\alignmentkey$}
        \Return -1
    \ElsIf{$t_1.\alignmentkey > t_2.\alignmentkey$}
        \Return 1
    \Else
        \Return 0
    \EndIf
\EndFunction
\end{algorithmic}
\end{algorithm}

%======================================================================
\chapter{Security Analysis}
%======================================================================

This chapter provides a formal security analysis of our oblivious multi-way band join algorithm. We prove that the algorithm's memory access patterns reveal no information about the input data beyond what is explicitly allowed (table sizes and tree structure). Our proof follows a modular approach, building from simple components to the complete algorithm using the composition theorem for oblivious operations.

%----------------------------------------------------------------------
\section{Security Model and Definitions}
%----------------------------------------------------------------------

We begin by formally defining oblivious operations and stating the composition theorem that underlies our security proof.

\subsection{Oblivious Operations}

\begin{definition}[Oblivious Operation]
An operation $\mathcal{O}: \mathcal{D} \rightarrow \mathcal{D}'$ is \emph{oblivious} if for any two input sequences $X, Y \in \mathcal{D}$ with $|X| = |Y|$, the access patterns $\mathcal{AP}(\mathcal{O}(X))$ and $\mathcal{AP}(\mathcal{O}(Y))$ are identically distributed.
\end{definition}

Intuitively, an oblivious operation accesses memory in a pattern that depends only on the size of the input, not on the actual data values. An adversary observing the memory accesses learns nothing about the data beyond its size.

\subsection{Composition Theorem}

The following theorem, standard in the oblivious algorithms literature, allows us to build complex oblivious algorithms from simple oblivious components:

\begin{theorem}[Sequential Composition]
\label{thm:composition}
If $\mathcal{O}_1: \mathcal{D} \rightarrow \mathcal{D}'$ and $\mathcal{O}_2: \mathcal{D}' \rightarrow \mathcal{D}''$ are oblivious operations, then their sequential composition $(\mathcal{O}_2 \circ \mathcal{O}_1): \mathcal{D} \rightarrow \mathcal{D}''$ defined by $(\mathcal{O}_2 \circ \mathcal{O}_1)(x) = \mathcal{O}_2(\mathcal{O}_1(x))$ is also oblivious.
\end{theorem}

\begin{proof}
For any inputs $x, y \in \mathcal{D}$ with $|x| = |y|$:
\begin{enumerate}
\item $\mathcal{AP}(\mathcal{O}_1(x)) \equiv \mathcal{AP}(\mathcal{O}_1(y))$ since $\mathcal{O}_1$ is oblivious
\item Let $x' = \mathcal{O}_1(x)$ and $y' = \mathcal{O}_1(y)$. Since $\mathcal{O}_1$ is oblivious, $|x'| = |y'|$
\item $\mathcal{AP}(\mathcal{O}_2(x')) \equiv \mathcal{AP}(\mathcal{O}_2(y'))$ since $\mathcal{O}_2$ is oblivious
\item Therefore, $\mathcal{AP}((\mathcal{O}_2 \circ \mathcal{O}_1)(x)) \equiv \mathcal{AP}((\mathcal{O}_2 \circ \mathcal{O}_1)(y))$
\end{enumerate}
Hence, $\mathcal{O}_2 \circ \mathcal{O}_1$ is oblivious.
\end{proof}

\subsection{Security Goal}

Our security goal is to prove the following theorem:

\begin{theorem}[Main Security Theorem]
\label{thm:main-security}
The oblivious multi-way band join algorithm is oblivious. That is, for any two sets of input tables with the same sizes, tree structure, and output size, the memory access patterns are identically distributed.
\end{theorem}

We prove this theorem through a hierarchical approach, starting with individual components and building up to the complete algorithm.

%----------------------------------------------------------------------
\section{Level 1: Base Component Security}
%----------------------------------------------------------------------

We first prove that our custom window functions, comparators, and update functions can be converted to oblivious implementations. Our conversion strategy relies on two key techniques:

\begin{enumerate}
\item \textbf{Arithmetic conversion}: Replace all conditional branches with arithmetic operations using 0/1 predicates. For any condition, we compute a predicate $p \in \{0,1\}$ and use multiplication: $\text{result} = p \cdot \text{value}_{\text{true}} + (1-p) \cdot \text{value}_{\text{false}}$.

\item \textbf{Access pattern uniformity}: Ensure all execution paths access the same memory locations in the same order, regardless of data values.
\end{enumerate}

This approach transforms data-dependent control flow into data-oblivious arithmetic operations, ensuring that the memory access pattern is independent of the input data values.

\subsection{Window Functions}

\begin{lemma}
\label{lem:window-local-sum}
\textsc{WindowComputeLocalSum} (Algorithm~\ref{alg:window-compute-local-sum}) can be converted to an oblivious implementation.
\end{lemma}

\begin{proof}
The function's conditional logic on entry type can be converted to oblivious form:
\begin{enumerate}
\item \textbf{Access pattern}: Always reads $\text{window}[0].\localsum$ and $\text{window}[1].\fieldtype$, $\text{window}[1].\localmult$, and always writes to $\text{window}[1].\localsum$.
\item \textbf{Arithmetic conversion}: The conditional branch becomes:
\begin{align}
\text{is\_source} &= (\text{window}[1].\fieldtype == \typesource) \in \{0,1\} \nonumber \\
\text{window}[1].\localsum &= \text{window}[0].\localsum \nonumber \\
&\quad + \text{is\_source} \cdot \text{window}[1].\localmult \nonumber
\end{align}
\end{enumerate}
This eliminates the conditional branch while preserving functionality: when SOURCE, adds $\localmult$; otherwise adds 0.
\end{proof}

\begin{lemma}
\label{lem:window-local-interval}
\textsc{WindowComputeLocalInterval} (Algorithm~\ref{alg:window-compute-local-interval}) can be converted to an oblivious implementation.
\end{lemma}

\begin{proof}
The function's conditional interval computation can be made oblivious:
\begin{enumerate}
\item \textbf{Access pattern}: Always read $\text{window}[0]$ and $\text{window}[1]$ fields, always write to $\text{window}[1].\localinterval$.
\item \textbf{Arithmetic conversion}: The conditional check becomes:
\begin{align}
\text{is\_pair} &= (\text{window}[0].\fieldtype == \typestart) \nonumber \\
&\quad \cdot (\text{window}[1].\fieldtype == \typeend) \in \{0,1\} \nonumber \\
\text{interval} &= \text{window}[1].\localsum - \text{window}[0].\localsum \nonumber \\
\text{window}[1].\localinterval &= \text{is\_pair} \cdot \text{interval} \nonumber \\
&\quad + (1 - \text{is\_pair}) \cdot \text{window}[1].\localinterval \nonumber
\end{align}
\end{enumerate}
The write always happens (either new interval or preserving existing value).
\end{proof}

\begin{lemma}
\label{lem:window-foreign-sum}
\textsc{WindowComputeForeignSum} (Algorithm~\ref{alg:window-compute-foreign-sum}) can be converted to an oblivious implementation.
\end{lemma}

\begin{proof}
The function's three-way branch can be converted to arithmetic operations:
\begin{enumerate}
\item \textbf{Access pattern}: Always read $\text{window}[0]$ fields and $\text{window}[1]$ fields, always write to $\text{window}[1].\localweight$ and $\text{window}[1].\foreigncumsum$.
\item \textbf{Arithmetic conversion}: The type-based branching becomes:
\begin{align}
\text{is\_start} &= (\text{window}[1].\fieldtype == \typestart) \in \{0,1\} \nonumber \\
\text{is\_end} &= (\text{window}[1].\fieldtype == \typeend) \in \{0,1\} \nonumber \\
\text{is\_source} &= (\text{window}[1].\fieldtype == \typesource) \in \{0,1\} \nonumber \\
\text{weight\_delta} &= \text{is\_start} \cdot \text{window}[1].\localmult \nonumber \\
&\quad - \text{is\_end} \cdot \text{window}[1].\localmult \nonumber \\
\text{window}[1].\localweight &= \text{window}[0].\localweight + \text{weight\_delta} \nonumber \\
\text{safe\_denom} &= \text{is\_source} \cdot \text{window}[0].\localweight + (1 - \text{is\_source}) \cdot 1 \nonumber \\
\text{foreign\_delta} &= \text{is\_source} \cdot (\text{window}[1].\finalmult / \text{safe\_denom}) \nonumber \\
\text{window}[1].\foreigncumsum &= \text{window}[0].\foreigncumsum + \text{foreign\_delta} \nonumber
\end{align}
\end{enumerate}
The safe denominator ensures division is never by zero: it uses the actual weight for SOURCE entries and 1 otherwise.
\end{proof}

\begin{lemma}
\label{lem:window-foreign-interval}
\textsc{WindowComputeForeignInterval} (Algorithm~\ref{alg:window-compute-foreign-interval}) can be converted to an oblivious implementation.
\end{lemma}

\begin{proof}
The function's conditional logic can be made oblivious:
\begin{enumerate}
\item \textbf{Access pattern}: Always read $\text{window}[0]$ and $\text{window}[1]$ fields, always write to $\text{window}[1].\foreigninterval$ and $\text{window}[1].\foreignsum$.
\item \textbf{Arithmetic conversion}: The conditional becomes:
\begin{align}
\text{is\_pair} &= (\text{window}[0].\fieldtype == \typestart) \nonumber \\
&\quad \cdot (\text{window}[1].\fieldtype == \typeend) \in \{0,1\} \nonumber \\
\text{interval} &= \text{window}[1].\foreigncumsum - \text{window}[0].\foreigncumsum \nonumber \\
\text{window}[1].\foreigninterval &= \text{is\_pair} \cdot \text{interval} \nonumber \\
&\quad + (1 - \text{is\_pair}) \cdot \text{window}[1].\foreigninterval \nonumber \\
\text{window}[1].\foreignsum &= \text{is\_pair} \cdot \text{window}[0].\foreigncumsum \nonumber \\
&\quad + (1 - \text{is\_pair}) \cdot \text{window}[1].\foreignsum \nonumber
\end{align}
\end{enumerate}
\end{proof}

\subsection{Comparators}

\begin{lemma}
\label{lem:comparator-join}
\textsc{ComparatorJoinAttr} (Algorithm~\ref{alg:comparator-join-attr}) can be converted to an oblivious implementation.
\end{lemma}

\begin{proof}
The comparator's conditional logic can be made oblivious:
\begin{enumerate}
\item \textbf{Access pattern}: Always read both elements' $\joinattr$, $\fieldtype$, and $\fieldequalitytype$ fields, and always access the precedence table.
\item \textbf{Arithmetic conversion}: Convert the nested conditionals to arithmetic:
\begin{align}
\text{cmp} &= \text{sign}(e_1.\joinattr - e_2.\joinattr) \in \{-1, 0, 1\} \nonumber \\
\text{is\_equal} &= (\text{cmp} == 0) \in \{0,1\} \nonumber \\
p_1 &= \text{GetPrecedence}(e_1.\fieldtype, e_1.\fieldequalitytype) \nonumber \\
p_2 &= \text{GetPrecedence}(e_2.\fieldtype, e_2.\fieldequalitytype) \nonumber \\
\text{prec\_cmp} &= \text{sign}(p_1 - p_2) \in \{-1, 0, 1\} \nonumber \\
\text{result} &= (1 - \text{is\_equal}) \cdot \text{cmp} + \text{is\_equal} \cdot \text{prec\_cmp} \nonumber
\end{align}
\end{enumerate}
The precedence lookup uses both type and equality type fields as indices.
\end{proof}

\begin{lemma}
\label{lem:comparator-pairwise}
\textsc{ComparatorPairwise} (Algorithm~\ref{alg:comparator-pairwise}) can be converted to an oblivious implementation.
\end{lemma}

\begin{proof}
The comparator has three-level comparison logic that can be made oblivious:
\begin{enumerate}
\item \textbf{Access pattern}: Always read both elements' $\fieldtype$ and $\fieldindex$ fields.
\item \textbf{Arithmetic conversion}: Convert the three-level priority system:
\begin{align}
\text{is\_target}_1 &= (e_1.\fieldtype \in \{\typestart, \typeend\}) \in \{0,1\} \nonumber \\
\text{is\_target}_2 &= (e_2.\fieldtype \in \{\typestart, \typeend\}) \in \{0,1\} \nonumber \\
\text{type\_priority} &= \text{is\_target}_2 - \text{is\_target}_1 \in \{-1, 0, 1\} \nonumber \\
\text{idx\_cmp} &= \text{sign}(e_1.\fieldindex - e_2.\fieldindex) \in \{-1, 0, 1\} \nonumber \\
\text{is\_start}_1 &= (e_1.\fieldtype == \typestart) \in \{0,1\} \nonumber \\
\text{is\_start}_2 &= (e_2.\fieldtype == \typestart) \in \{0,1\} \nonumber \\
\text{start\_first} &= \text{is\_start}_1 - \text{is\_start}_2 \in \{-1, 0, 1\} \nonumber \\
\text{same\_priority} &= (\text{type\_priority} == 0) \in \{0,1\} \nonumber \\
\text{same\_index} &= (\text{idx\_cmp} == 0) \in \{0,1\} \nonumber \\
\text{result} &= (1 - \text{same\_priority}) \cdot \text{type\_priority} \nonumber \\
&\quad + \text{same\_priority} \cdot (1 - \text{same\_index}) \cdot \text{idx\_cmp} \nonumber \\
&\quad + \text{same\_priority} \cdot \text{same\_index} \cdot \text{start\_first} \nonumber
\end{align}
\end{enumerate}
Priority order: (1) Target entries before SOURCE, (2) by original index, (3) START before END.
\end{proof}

\subsection{Update Functions}

\begin{lemma}
\label{lem:update-target-mult}
\textsc{UpdateTargetMultiplicity} (Algorithm~\ref{alg:update-target-multiplicity}) is inherently oblivious.
\end{lemma}

\begin{proof}
The function performs pure arithmetic:
\begin{enumerate}
\item \textbf{Access pattern}: Always read from both $t$ and $e$, always write to $t.\localmult$.
\item \textbf{No conversion needed}: The multiplication $t.\localmult \times e.\localinterval$ is already oblivious.
\end{enumerate}
\end{proof}

\begin{lemma}
\label{lem:update-target-final}
\textsc{UpdateTargetFinalMultiplicity} (Algorithm~\ref{alg:update-target-final}) is inherently oblivious.
\end{lemma}

\begin{proof}
The function performs pure arithmetic:
\begin{enumerate}
\item \textbf{Access pattern}: Always read $e.\foreigninterval$, $e.\foreignsum$, and $t.\localmult$, always write to $t.\finalmult$ and $t.\foreignsum$.
\item \textbf{No conversion needed}: The operations $t.\finalmult = e.\foreigninterval \times t.\localmult$ and $t.\foreignsum = e.\foreignsum$ are pure arithmetic/assignment.
\end{enumerate}
\end{proof}

%----------------------------------------------------------------------
\section{Level 2: Composed Operation Security}
%----------------------------------------------------------------------

Having shown that our base components can be converted to oblivious implementations, we now prove that composing these converted oblivious versions with established oblivious primitives yields oblivious operations.

\subsection{Oblivious Primitives}

We rely on the following well-established oblivious primitives:

\begin{assumption}
\label{assum:primitives}
The following operations are oblivious:
\begin{itemize}
\item \textsc{ObliviousSort}: Uses Batcher's bitonic sort~\cite{batcher1968} with a fixed comparison network
\item \textsc{ObliviousExpand}: From ODBJ~\cite{krastnikov2020}, expands tables obliviously
\item \textsc{LinearPass}: Iterates through a table with fixed window size 2
\item \textsc{ParallelPass}: Applies a function to each element independently
\item \textsc{Map}: Transforms each element independently
\end{itemize}
\end{assumption}

\subsection{Composed Operations}

\begin{lemma}
\label{lem:sort-with-comparator}
For any comparator $C$ that can be converted to oblivious form, \textsc{ObliviousSort}$(T, C_{oblivious})$ is oblivious.
\end{lemma}

\begin{proof}
By Assumption~\ref{assum:primitives}, \textsc{ObliviousSort} has a fixed comparison pattern based only on table size. By Lemmas~\ref{lem:comparator-join}-\ref{lem:comparator-pairwise}, our comparators can be converted to oblivious implementations. Using the converted oblivious versions $C_{oblivious}$ and applying Theorem~\ref{thm:composition}, the composition is oblivious.\end{proof}

\begin{lemma}
\label{lem:linear-with-window}
For any window function $W$ that can be converted to oblivious form, \textsc{LinearPass}$(T, W_{oblivious})$ is oblivious.
\end{lemma}

\begin{proof}
\textsc{LinearPass} has a deterministic iteration pattern based only on table size (with fixed window size 2). By Lemmas~\ref{lem:window-local-sum}-\ref{lem:window-foreign-interval}, our window functions can be converted to oblivious implementations. Using the converted versions $W_{oblivious}$ and applying Theorem~\ref{thm:composition}, the composition is oblivious.\end{proof}

\begin{lemma}
\label{lem:parallel-with-update}
For any update function $U$ that is inherently oblivious or can be converted to oblivious form, \textsc{ParallelPass}$(T, U_{oblivious})$ is oblivious.
\end{lemma}

\begin{proof}
\textsc{ParallelPass} applies $U$ to each element independently with a fixed access pattern. By Lemmas~\ref{lem:update-target-mult}-\ref{lem:update-target-final}, our update functions are inherently oblivious (pure arithmetic). The parallel application maintains obliviousness.\end{proof}

%----------------------------------------------------------------------
\section{Level 3: Phase Security}
%----------------------------------------------------------------------

We prove that each phase of our algorithm is oblivious.

\subsection{Initialization Phase}

\begin{lemma}
\label{lem:init-oblivious}
The Initialization phase (Algorithm~\ref{alg:initialize}) is oblivious.
\end{lemma}

\begin{proof}
Initialization consists of:
\begin{enumerate}
\item \textsc{Map} to add metadata columns
\item \textsc{LinearPass} with \textsc{WindowSetOriginalIndex} (Algorithm~\ref{alg:window-set-orig-index})
\end{enumerate}
Both operations access each element exactly once in a predetermined order. By Theorem~\ref{thm:composition}, their composition is oblivious.\end{proof}

\subsection{Bottom-Up Phase}

\begin{lemma}
\label{lem:bottom-up-oblivious}
The Bottom-Up phase is oblivious.
\end{lemma}

\begin{proof}
For each node in post-order (public tree structure), the phase performs (Algorithm~\ref{alg:compute-local}):
\begin{align}
\text{BottomUp} = &\text{ CombineTable (Algorithm~\ref{alg:combine-table})} \nonumber \\
&\rightarrow \text{ObliviousSort(ComparatorJoinAttr)} \nonumber \\
&\rightarrow \text{LinearPass(WindowComputeLocalSum)} \nonumber \\
&\rightarrow \text{ObliviousSort(ComparatorPairwise)} \nonumber \\
&\rightarrow \text{LinearPass(WindowComputeLocalInterval)} \nonumber \\
&\rightarrow \text{ObliviousSort(ComparatorEndFirst)} \nonumber \\
&\rightarrow \text{ParallelPass(UpdateTargetMultiplicity)} \nonumber
\end{align}

Each operation is oblivious by Lemmas~\ref{lem:sort-with-comparator}-\ref{lem:parallel-with-update}. The number of iterations depends only on the public tree structure. By repeated application of Theorem~\ref{thm:composition}, the entire phase is oblivious.\end{proof}

\subsection{Top-Down Phase}

\begin{lemma}
\label{lem:top-down-oblivious}
The Top-Down phase is oblivious.
\end{lemma}

\begin{proof}
The structure mirrors the Bottom-Up phase but with pre-order traversal and different window/update functions (Algorithm~\ref{alg:propagate-final}). Each component operation is oblivious by the same arguments. By Theorem~\ref{thm:composition}, the phase is oblivious.\end{proof}

\subsection{Distribution and Expansion Phase}

\begin{lemma}
\label{lem:expand-oblivious}
The Distribution and Expansion phase is oblivious.
\end{lemma}

\begin{proof}
This phase applies \textsc{ObliviousExpand} to each table. By Assumption~\ref{assum:primitives}, \textsc{ObliviousExpand} is oblivious. The operation is applied to each table independently based on the public tree structure.\end{proof}

\subsection{Alignment and Concatenation Phase}

\begin{lemma}
\label{lem:align-oblivious}
The Alignment and Concatenation phase is oblivious.
\end{lemma}

\begin{proof}
For each parent-child pair, the phase performs:
\begin{enumerate}
\item \textsc{ObliviousSort} on parent table
\item \textsc{ParallelPass} to compute alignment keys
\item \textsc{ObliviousSort} on child table
\item \textsc{HorizontalConcatenate}
\end{enumerate}
Each operation is oblivious, and their composition is oblivious by Theorem~\ref{thm:composition}.\end{proof}

%----------------------------------------------------------------------
\section{Level 4: Complete Algorithm Security}
%----------------------------------------------------------------------

We now prove our main security theorem.

\begin{proof}[Proof of Theorem~\ref{thm:main-security}]
The complete algorithm performs:
\begin{align}
\text{Algorithm} = &\text{ Initialization} \nonumber \\
&\rightarrow \text{Bottom-Up Phase} \nonumber \\
&\rightarrow \text{Top-Down Phase} \nonumber \\
&\rightarrow \text{Distribution \& Expansion} \nonumber \\
&\rightarrow \text{Alignment \& Concatenation} \nonumber
\end{align}

By Lemmas~\ref{lem:init-oblivious}, \ref{lem:bottom-up-oblivious}, \ref{lem:top-down-oblivious}, \ref{lem:expand-oblivious}, and \ref{lem:align-oblivious}, each phase is oblivious.

By repeated application of Theorem~\ref{thm:composition} (sequential composition), the complete algorithm is oblivious.

Therefore, for any two sets of input tables with the same sizes, tree structure, and output size, the memory access patterns are identically distributed, revealing no information about the actual data values, join selectivities, or which tuples match.\end{proof}

%----------------------------------------------------------------------
\section{Memory Access Pattern Analysis}
%----------------------------------------------------------------------

\textit{[This section is reserved for detailed analysis of memory access patterns and will be completed in future work.]}

%----------------------------------------------------------------------
\section{Summary}
%----------------------------------------------------------------------

We have proven that our oblivious multi-way band join algorithm maintains complete data obliviousness through a modular security proof. The proof builds from simple oblivious components (window functions, comparators, update functions) through composed operations and phases, ultimately establishing that the complete algorithm reveals no information through its memory access patterns beyond what is explicitly allowed (table sizes and tree structure).

The security guarantee holds even for band joins with inequality constraints, where the number of matching tuples and the distribution of values within ranges remain completely hidden from any adversary observing the execution.

% %======================================================================
\chapter{Implementation}
%======================================================================

% Placeholder chapter - to be completed % Removed - implementation details integrated into algorithm chapters

%======================================================================
\chapter{Evaluation}
%======================================================================

We evaluate our oblivious multi-way band join algorithm by comparing its performance against OJOIN~\cite{hu2025optimal}, the state-of-the-art oblivious multi-way join algorithm. Our experiments use the same TPC-H benchmark setup to ensure a fair comparison, focusing on both multi-way equality joins and band joins with inequality constraints.

%----------------------------------------------------------------------
\section{Implementation}
%----------------------------------------------------------------------

We implemented our algorithm in C++ for Intel SGX2.

\subsection{Data Preprocessing}

To simplify the implementation and focus on the core algorithmic performance, we preprocess the TPC-H tables as follows:

\begin{itemize}
\item \textbf{Type conversion}: All date and string fields are converted to integers. Dates are represented as days since a reference date (1970-01-01), while strings are mapped to integer identifiers through a preprocessing dictionary.

\item \textbf{Table duplication}: When a query requires using the same table multiple times (self-joins or multiple references), we create distinct copies with appropriate renaming. This simplifies the tree structure handling without affecting the algorithm's complexity.

\item \textbf{Memory layout}: Tables are stored as arrays of row objects within the enclave. This row-oriented storage is a source of performance overhead compared to column-oriented formats that would better match our sequential access patterns.
\end{itemize}


%----------------------------------------------------------------------
\section{Experimental Setup}
%----------------------------------------------------------------------

\subsection{Dataset and Queries}

We use the TPC-H benchmark with scale factor 0.1 (SF=0.1), matching the setup used in the OJOIN~\cite{hu2025optimal} evaluation. This generates approximately 100MB of raw data across eight tables. We evaluate pure \texttt{SELECT-FROM-WHERE} queries without subqueries or aggregation operations, focusing on the core join performance. The queries from the OJOIN~\cite{hu2025optimal} paper are:

\textbf{Multi-way Equality Joins (TM series):}
\begin{itemize}
\item \textbf{TM1}: 3-way join between \texttt{lineitem}, \texttt{orders}, and \texttt{customer}
\item \textbf{TM2}: 4-way join between \texttt{lineitem}, \texttt{orders}, \texttt{customer}, and \texttt{nation}
\item \textbf{TM3}: 5-way join adding \texttt{region} to TM2
\end{itemize}

\textbf{Band Joins (TB series):}
\begin{itemize}
\item \textbf{TB1}: Band join query with date range constraints on \texttt{lineitem}
\item \textbf{TB2}: Extended band join with overlapping date ranges across \texttt{lineitem} and \texttt{orders}
\end{itemize}

All queries follow the standard SQL pattern without \texttt{GROUP BY}, \texttt{HAVING}, subqueries, or aggregate functions. This allows us to focus purely on the join algorithm performance without the complexity of aggregation processing.

The specific band join queries tested are:

\textbf{TB1 Query:}
\begin{verbatim}
SELECT * FROM lineitem, orders
WHERE l_orderkey = o_orderkey
  AND l_shipdate >= o_orderdate
  AND l_shipdate <= o_orderdate + 30
\end{verbatim}

\textbf{TB2 Query:}
\begin{verbatim}
SELECT * FROM lineitem l1, lineitem l2, orders
WHERE l1.l_orderkey = o_orderkey
  AND l2.l_orderkey = o_orderkey
  AND l1.l_shipdate >= o_orderdate
  AND l1.l_shipdate <= o_orderdate + 30
  AND l2.l_shipdate >= o_orderdate + 15
  AND l2.l_shipdate <= o_orderdate + 45
\end{verbatim}

These queries test band join performance with date range constraints, where TB1 performs a single band join and TB2 involves overlapping date ranges across multiple joins.

\subsection{Hardware Configuration}

Experiments were conducted on a server with the following specifications:
\begin{itemize}
\item Intel Xeon E-2374G processor @ 3.70GHz with SGX support
\item 4 cores, 8 threads with AVX-512 support
\item 125GB RAM (120GB available)
\item Ubuntu 22.04.4 LTS with Linux kernel 5.15.0
\item NVMe SSD storage
\end{itemize}

\subsection{Metrics}

We measure the total execution time for each query, comparing our algorithm's runtime against OJOIN~\cite{hu2025optimal}.

%----------------------------------------------------------------------
\section{Results: Multi-way Equality Joins}
%----------------------------------------------------------------------

Table~\ref{tab:equality-joins} shows the performance comparison between our algorithm and OJOIN~\cite{hu2025optimal} for multi-way equality joins.

\begin{table}[h]
\centering
\caption{Performance comparison for multi-way equality joins}
\label{tab:equality-joins}
\begin{tabular}{llrrrr}
\toprule
Query & Scale Factor & Output Size & OJOIN (s) & Ours (s) & Speedup \\
\midrule
TM2 & 0.001 & 292 & -- & 0.77 & -- \\
TM2 & 0.01 & 29,929 & 10 & 10.38 & 0.96× \\
TM2 & 0.1 & 2,999,594 & 100 & OOM\footnotemark & -- \\
\bottomrule
\end{tabular}
\end{table}
\footnotetext{Out of memory error due to large output size (3M rows) exceeding enclave heap limit}

For TM2 at scale factor 0.01, our algorithm performs comparably to OJOIN~\cite{hu2025optimal} with both completing in approximately 10 seconds. However, at SF=0.1, the output size of nearly 3 million rows exceeds our current 512MB heap limit, while OJOIN completes in 100 seconds.


%----------------------------------------------------------------------
\section{Results: Band Joins}
%----------------------------------------------------------------------

Table~\ref{tab:band-joins} presents the results for band join queries with inequality constraints.

\begin{table}[h]
\centering
\caption{Performance comparison for band joins at different scale factors}
\label{tab:band-joins}
\begin{tabular}{llrrrr}
\toprule
Query & Scale Factor & Output Size & OJOIN (s) & Ours (s) & Speedup \\
\midrule
TB1 & 0.001 & 10 & -- & 2.58 & -- \\
TB1 & 0.01 & 111 & -- & 2.59 & -- \\
TB1 & 0.1 & 2,042 & 100 & 2.95 & 33.9× \\
\midrule
TB2 & 0.001 & 200 & -- & 2.61 & -- \\
TB2 & 0.01 & 4,002 & -- & 3.50 & -- \\
TB2 & 0.1 & 397,380 & 100,000 & OOM\footnotemark & -- \\
\bottomrule
\end{tabular}
\end{table}
\footnotetext{Out of memory error due to large output size (397K rows) exceeding enclave heap limit}

Our dual-entry technique provides significant advantages for band joins. For TB1 at scale factor 0.1, we achieve a remarkable 33.9× speedup over OJOIN~\cite{hu2025optimal}, completing the query in 2.95 seconds compared to OJOIN's 100 seconds. The performance remains consistent across different scale factors, demonstrating the efficiency of our approach.

For TB2, while we successfully process smaller scale factors, the SF=0.1 dataset produces 397,380 output rows which exceeds our current enclave heap limit of 512MB. OJOIN requires 100,000 seconds (over 27 hours) for this query, highlighting the computational challenge of large band joins.



%----------------------------------------------------------------------
\section{Discussion}
%----------------------------------------------------------------------

\subsection{Key Findings}

Our evaluation demonstrates:

\begin{enumerate}
\item \textbf{Exceptional band join performance}: TB1 shows 33.9× speedup over OJOIN~\cite{hu2025optimal} at SF=0.1
\item \textbf{Competitive equality joins}: TM2 performs comparably to OJOIN at SF=0.01
\item \textbf{High SGX overhead on small data}: TB1 runtime remains nearly constant (2.58-2.95s) across scale factors due to SGX enclave overhead dominating on smaller datasets
\item \textbf{Memory limitations}: Large output sizes (TB2: 397K rows, TM2: 3M rows) exceed current 512MB heap limit
\end{enumerate}

\subsection{Future Improvements}

\begin{itemize}
\item \textbf{Selective SGX execution}: Execute only the critical oblivious operations (window functions and comparators) inside SGX while keeping data outside the enclave. This would reduce memory requirements to constant space, enabling processing of arbitrarily large inputs and outputs.
\item \textbf{Columnar storage}: Transition from row-oriented to column-oriented storage for better cache utilization and sequential access patterns.
\item \textbf{Streaming preprocessing}: Integrate type conversion and table preparation into the main algorithm to avoid separate preprocessing passes.
\item \textbf{Cyclic query support}: Extend the algorithm to handle cyclic join graphs using generalized hypertree decomposition.
\end{itemize}

%----------------------------------------------------------------------
\section{Summary}
%----------------------------------------------------------------------

Our oblivious multi-way band join algorithm demonstrates substantial performance improvements over the state-of-the-art OJOIN~\cite{hu2025optimal} algorithm, particularly for band join queries. For TB1 at scale factor 0.1, we achieve a remarkable 33.9× speedup, reducing execution time from 100 seconds to just 2.95 seconds. For equality joins like TM2, we achieve comparable performance at smaller scale factors. The dual-entry technique proves highly effective for handling inequality constraints while maintaining complete data obliviousness. While memory constraints limit processing of very large result sets (millions of output rows), our implementation in SGX demonstrates the practicality of our approach for secure database applications.

%======================================================================
\chapter{Conclusion}
%======================================================================

This thesis presented the first oblivious algorithm for multi-way band joins, addressing a critical gap in secure database query processing. By adapting the classical Yannakakis algorithm to the oblivious computation model and introducing novel techniques for handling inequality constraints, we achieved both theoretical elegance and practical performance improvements.

\section{Summary of Contributions}

Our work makes three primary contributions to the field of oblivious database algorithms:

\textbf{1. Oblivious Adaptation of Yannakakis Algorithm}: We successfully transformed the classical Yannakakis algorithm for acyclic joins into a fully oblivious version. This required careful redesign of the two-phase semi-join approach, replacing data-dependent operations with oblivious primitives while preserving the algorithm's optimal complexity. Our adaptation maintains the algorithm's elegance while ensuring that memory access patterns reveal nothing about the input data.

\textbf{2. Dual-Entry Technique for Band Joins}: We introduced a novel dual-entry approach that enables efficient processing of inequality constraints in an oblivious manner. By creating START and END entries for range boundaries and using window-based computation, we can handle band joins without the exponential blowup that would result from naive approaches. This technique proved particularly effective, achieving a 33.9× speedup over OJOIN for the TB1 query.

\textbf{3. Rigorous Security Analysis}: We provided a comprehensive four-level security proof, demonstrating that our algorithm maintains complete data obliviousness. Starting from base components (window functions, comparators) and building up through composed operations, phases, and the complete algorithm, we proved that all memory access patterns are independent of input data values.

\section{Experimental Validation}

Our implementation in Intel SGX demonstrated the practicality of our approach:

\begin{itemize}
\item For band join query TB1 at scale factor 0.1, we achieved execution in 2.95 seconds compared to OJOIN's 100 seconds—a 33.9× improvement
\item For equality join TM2 at scale factor 0.01, we achieved comparable performance to OJOIN (approximately 10 seconds)
\end{itemize}

These results validate that sophisticated algorithmic techniques can overcome the performance penalties typically associated with oblivious computation.

\section{Practical Implications}

This work contributes to oblivious database research in several ways:

\textbf{Practical Secure Analytics}: By dramatically improving band join performance, we make secure processing of temporal and range queries practical for real applications. This is crucial for privacy-preserving analytics on sensitive data such as medical records or financial transactions.

\textbf{Dual-Entry Technique}: The dual-entry approach provides a new method for handling inequality constraints obliviously, which could be useful for other researchers working on similar problems.

\textbf{Implementation Experience}: Our SGX implementation provides insights into the practical challenges of deploying oblivious algorithms in secure hardware, including memory constraints and performance overheads.

\section{Future Directions}

Several promising avenues extend from this work:

\textbf{Complete SQL Engine}: Implementing a full oblivious SQL engine that supports GROUP BY, aggregation functions, and subqueries would enable processing of more complex analytical queries beyond simple joins.

\textbf{Columnar Memory Layout}: Transitioning from the current row-oriented storage to a columnar format would improve cache utilization and enable more efficient sequential access patterns, potentially providing significant performance gains.

\textbf{Extending to Cyclic Queries}: While our current algorithm handles acyclic join trees, many real queries involve cycles. Extending our techniques to work with generalized hypertree decompositions would broaden applicability.

\section{Closing Remarks}

This thesis addressed the specific problem of performing multi-way band joins obliviously, demonstrating that adapting classical algorithms to secure computation models can yield significant performance improvements. The 33.9× speedup achieved for TB1 queries shows that oblivious computation does not necessarily require accepting poor performance.

While memory constraints currently limit the size of queries we can process, the path forward is clear: selective execution of only the critical oblivious operations within SGX would enable handling of much larger datasets. Our work provides a foundation for continued research in oblivious database algorithms, contributing one piece to the larger puzzle of building practical privacy-preserving systems.

%----------------------------------------------------------------------
% END MATERIAL
% Bibliography, Appendices, Index, etc.
%----------------------------------------------------------------------

% Bibliography

% The following statement selects the style to use for references.  
% It controls the sort order of the entries in the bibliography and also the formatting for the in-text labels.
\bibliographystyle{plain}
% This specifies the location of the file containing the bibliographic information.  
% It assumes you're using BibTeX to manage your references (if not, why not?).
\cleardoublepage % This is needed if the "book" document class is used, to place the anchor in the correct page, because the bibliography will start on its own page.
% Use \clearpage instead if the document class uses the "oneside" argument
\phantomsection  % With hyperref package, enables hyperlinking from the table of contents to bibliography             
% The following statement causes the title "References" to be used for the bibliography section:
\renewcommand*{\bibname}{References}

% Add the References to the Table of Contents
\addcontentsline{toc}{chapter}{\textbf{References}}

\bibliography{uw-ethesis.bib}
% Tip: You can create multiple .bib files to organize your references. 
% Just list them all in the \bibliogaphy command, separated by commas (no spaces).

% Only include references that are actually cited in the text
%----------------------------------------------------------------------

% Appendices removed

% GLOSSARIES (Lists of definitions, abbreviations, symbols, etc. provided by the glossaries-extra package)
% -----------------------------
\printglossary
\cleardoublepage
\phantomsection		% allows hyperref to link to the correct page

%----------------------------------------------------------------------
\end{document} % end of logical document
