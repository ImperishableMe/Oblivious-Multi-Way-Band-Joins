% ==============================================================================
% NOTATION AND SYMBOL DEFINITIONS
% ==============================================================================
% This file contains all mathematical notation and symbol definitions used
% throughout the thesis. It includes both LaTeX macros and glossary entries.
%
% USAGE CONVENTION:
% - Paragraphs/Text: "name (symbol)" first time, then "name + symbol" after
% - Equations: symbol only (e.g., $\localmult$, $\bigO{...}$)
% - Tables: same as paragraphs (name + symbol format)
%
% Examples:
% - First mention: "local multiplicity ($\localmult$)"
% - Later: "local multiplicity $\localmult$" or "the local multiplicity increases"
% - Equations: "$\localmult = \sum_{i} \alpha_i$"

% ==============================================================================
% LATEX MACROS FOR MATHEMATICAL NOTATION
% ==============================================================================

% ------------------------------------------------------------------------------
% Basic Size and Complexity Notation
% ------------------------------------------------------------------------------
\newcommand{\inputsize}{N} % Total input size
\newcommand{\outputsize}{\text{OUT}} % Output size
\newcommand{\tablesize}{n} % Size of a single table
\newcommand{\numtables}{k} % Number of tables
\newcommand{\bigO}[1]{O(#1)} % Big O notation

% ------------------------------------------------------------------------------
% Algorithm-Specific Notation
% ------------------------------------------------------------------------------
\newcommand{\localmult}{\alpha_{\text{local}}} % Local multiplicity
\newcommand{\finalmult}{\alpha_{\text{final}}} % Final multiplicity
\newcommand{\foreignmult}{\alpha_{\text{foreign}}} % Foreign multiplicity
\newcommand{\cumsum}{C} % Cumulative sum counter

% ------------------------------------------------------------------------------
% Algorithm Variables
% ------------------------------------------------------------------------------
\newcommand{\foreigncumsum}{C_{\text{foreign}}} % Foreign cumulative sum
\newcommand{\localweight}{w_{\text{local}}} % Local weight counter
\newcommand{\localcumsum}{C_{\text{local}}} % Local cumulative sum
\newcommand{\entry}{e} % General entry variable
\newcommand{\startentry}{e_s} % Start entry variable (avoid conflict with constants)
\newcommand{\stopentry}{e_t} % End/terminal entry variable (avoid conflict with constants)
\newcommand{\joinattr}{\text{join\_attr}} % Join attribute

% ------------------------------------------------------------------------------
% Field Accessor Notation
% ------------------------------------------------------------------------------
\newcommand{\fieldtype}{\text{type}} % Entry type field (replaces .type)
\newcommand{\fieldequalitytype}{\text{equality}} % Equality type field
\newcommand{\fielddata}{d} % Entry data/tuple reference field (replaces .data)
\newcommand{\fieldindex}{\text{orig\_idx}} % Original index field

% ------------------------------------------------------------------------------
% Counter Variables  
% ------------------------------------------------------------------------------
\newcommand{\foreignsum}{F_{\text{sum}}} % Foreign sum

% ------------------------------------------------------------------------------
% Auxiliary Variables
% ------------------------------------------------------------------------------
\newcommand{\precedence}{\pi} % Precedence mapping

% ------------------------------------------------------------------------------
% Predefined Complexity Expressions
% ------------------------------------------------------------------------------
\newcommand{\sortcomplexity}{\bigO{\tablesize \log^2 \tablesize}} % Oblivious sorting complexity

% ------------------------------------------------------------------------------
% Algorithm Names and Formatting
% ------------------------------------------------------------------------------
\newcommand{\odbj}{\textsc{odbj}} % ODBJ algorithm name

% ------------------------------------------------------------------------------
% Entry Type Constants
% ------------------------------------------------------------------------------
\newcommand{\typesource}{\text{SOURCE}} % SOURCE entry type constant
\newcommand{\typestart}{\text{START}} % TARGET_START entry type constant
\newcommand{\typeend}{\text{END}} % TARGET_END entry type constant


% ------------------------------------------------------------------------------
% Equality Type Constants
% ------------------------------------------------------------------------------
\newcommand{\equalityequal}{\text{EQ}} % Equal equality type constant
\newcommand{\equalitynonequal}{\text{NEQ}} % Non-equal equality type constant

% ------------------------------------------------------------------------------
% Boundary Parameter Variables
% ------------------------------------------------------------------------------
\newcommand{\deviationone}{d_1} % First boundary deviation
\newcommand{\deviationtwo}{d_2} % Second boundary deviation  
\newcommand{\equalityone}{eq_1} % First boundary equality type
\newcommand{\equalitytwo}{eq_2} % Second boundary equality type
\newcommand{\constraint}{\mathcal{C}} % Constraint function
\newcommand{\constraintparam}{\theta} % Constraint parameters: ((d1, eq1), (d2, eq2))

% ------------------------------------------------------------------------------
% Table Type Terminology (following Krastnikov's ODBJ conventions)
% ------------------------------------------------------------------------------
\newcommand{\inputtable}{input table} % Original unmodified table
\newcommand{\inputtables}{input tables} % Plural form
\newcommand{\augmentedtable}{augmented table} % Table extended with multiplicity metadata
\newcommand{\augmentedtables}{augmented tables} % Plural form
\newcommand{\combinedtable}{combined table} % Array of entries for dual-entry processing
\newcommand{\combinedtables}{combined tables} % Plural form
\newcommand{\expandedtable}{expanded table} % Table where each tuple appears finalmult times
\newcommand{\expandedtables}{expanded tables} % Plural form
\newcommand{\alignedtable}{aligned table} % Expanded table reordered for concatenation
\newcommand{\alignedtables}{aligned tables} % Plural form

% ------------------------------------------------------------------------------
% Table Variable Notation (consistent macro usage)
% ------------------------------------------------------------------------------
\newcommand{\Rcomb}{R_{\text{comb}}} % Combined table variable
\newcommand{\Rtarget}{R_{\text{target}}} % Target table variable  
\newcommand{\Rsource}{R_{\text{source}}} % Source table variable
\newcommand{\Rtruncated}{R_{\text{truncated}}} % Truncated table variable
\newcommand{\Rv}{R_v} % Table at node v
\newcommand{\Rc}{R_c} % Table at child node c

% ------------------------------------------------------------------------------
% Temporary Metadata Field Accessors
% ------------------------------------------------------------------------------
\newcommand{\localsum}{\text{local\_sum}} % Local sum field for cumulative computation
\newcommand{\localinterval}{\text{local\_interval}} % Local interval field for range computation
\newcommand{\foreigninterval}{\text{foreign\_interval}} % Foreign interval field for range computation
\newcommand{\copyindex}{\text{copy\_index}} % Index of copy among all copies of same tuple
\newcommand{\alignmentkey}{\text{alignment\_key}} % Computed alignment position for sorting

% ==============================================================================
% GLOSSARY ENTRIES FOR LIST OF SYMBOLS
% ==============================================================================
% Note: Glossary entries are defined in uw-ethesis.tex after packages are loaded
% to avoid undefined command errors. Only LaTeX macros are defined here.